% Document notes:
% Started messing around manually with the packages to update circuitikz to get a curved capacitor. Apparently was running a version earlier than 2019, even though got the package in 2020. Was dumb. Just use texhash next time; it updates "the database" automatically.
% See note on first circuit (not able to put nodes on switch B)
% Keep getting the warning: "LaTeX Font Warning: Font shape `OT1/cmr/m/n' in size <4.2> not available (Font) size <5> substituted on input line 87." even though I have not used another other font. Incredibly irritating. See log.
% Everything kept breaking with this lab report. It's probably the worst one besides there being no overfull bad boxes.
% Fix the "noindents". Put it as a global setting instead of scattering the code with it.
\documentclass [12pt, letterpaper, twoside] {article}
\usepackage[utf8]{inputenc}
\usepackage [left=1.0in, right=1.0in, top=1.0in, bottom=1.0in] {geometry}
% For updated time
\usepackage {datetime}
% For drawing pictures
\usepackage {tikz}
% For equations
\usepackage {amsmath}
% To make tables
\usepackage {tabu}
% For multiple rows in table slot
\usepackage {multirow}
% For comments, I think
\usepackage {verbatim}
% To add captions
\usepackage {caption}
\usepackage {float}
% To make graphs
\usepackage {pgfplots}
% To make scatter plots
\usepackage{pgfplotstable}
% To make circuit drawings
\usepackage{circuitikz}
%\usepackage[siunitx, RPvoltages]{circuitikzgit}
% To make tables side by side
\usepackage{subfig}
% To make inline fractions
\usepackage{xfrac}

\raggedbottom
\pgfplotsset{compat=1.16}
\begin {document}
\begin {titlepage}
\begin {center}
Department of Biological, Chemical, and Physical Science\\
\vspace {0.1cm}
Illinois Institute of Technology\\
\vspace {0.1cm}
General Physics II: Electromagnetism (PHYS 221-01)\\
\vspace* {\fill}
\begingroup
\Large
\textbf {RC Circuits}
\vspace {0.35cm}

\normalsize
Lab 7 
\vspace {1.5cm}
\endgroup
\vspace* {\fill}
\end {center}

\vspace*{\fill}
\begin {flushright}
\footnotesize
Emily Pang, Lavanya Roy (lab partner) \\
Date of experiment: 26 Feb 2020 \\
Due date: 1 Apr 2020 \\
Lab section L06 \\
TA: Will Limestall \\
Updated \usdate\today~(\currenttime)
\end {flushright}
\end {titlepage}
\subsection* {STATEMENT OF OBJECTIVE}
The objective of this lab was to examine the relationship between capacitance, total resistance, and time in a circuit by charging or discharging a capacitor.

\subsection* {THEORY}
The voltage across a capacitor depends on the max voltage it could be charged to, as well as the time it has been charging and a time constant involving the capacitance and the total resistance in the circuit. For charging a capacitor, the voltage can be expressed as:
\begin{equation}
  \begin{split}
    V(t) &= V_{\text{0}}\left(1-e^{-\sfrac{t}{\tau}}\right) \\
  \end{split}
\end{equation}
where \(V_{\text{0}}\) is the "maximum" (not possible to reach this value) voltage the capacitor can charge to, \(t\) is the time the capacitor has been charging, and \(\tau\) is the product of the total resistance in the circuit and the capacitance of the capacitor.

\noindent
By the same logic, the voltage across a capacitor when it is discharging can be expressed as:
\begin{equation}
  \begin{split}
    V(t) &= V_{0}e^{-\sfrac{t}{\tau}} \\
  \end{split}
\end{equation}
where \(V_{0}\) is the voltage the capacitor starts at, \(t\) is the time the capacitor has been discharging, and \(\tau\) is the product of the total resistance in the circuit and the capacitance of the capacitor.

\subsection* {EQUIPMENT}
  \noindent
  \begin {itemize}
    \itemsep0em
    \item {one breadboard}
    \item {one multimeter}
    \item {one capacimeter}
    \item {one digital oscilloscope}
    \item {one function generator}
    \item {one PASCO Capstone Sensor and associated softward}
    \item {wires to connect power supply and breadboard}
    \item {safety glasses}
  \end {itemize}

\subsection* {PROCEDURE}
There were two types of time constants measured in this lab. Of the two capacitors given, the large capacitor was used for the first part of the lab. Its capacitance was measured and recorded. The circuit in Figure 1 was then constructed using the large capacitor.

\noindent
For the second part of the lab, the small capacitor and the oscilloscope were used and Circuit b of Figure 1 was constructed. Using the oscilloscope, \(\tau\) was again calculated using milliseconds instead of seconds and observing the function behavior as the capacitor was charged and discharged.

\begin{figure}
  \centering
  \subfloat[]{
    \begin{circuitikz}[american voltages,label/.style={red}]
      \draw (0.75,2.75) node[label]{\bf{A}};
      \draw (1.25,1.5) node[label]{\bf{B}};
      \draw (0.3,1.9) node {\tiny{+}};
      \draw (0.3,1.15) node {\tiny{-}};
      \draw (0,3) to [battery] (0,0);
      % For some reason, able to get nodes on switch A but not on switch B (nodes are placed at the branch out above and below)
      \draw (0,3) -- (0.5,3) to [normal open switch,*-*] (1,3) -- (1.5,3) to [normal open switch] (1.5,0) -- (0,0);
      \draw (1.5,3) -- (3,3) to [resistor,l_=R] (3,1.5) to [cC,l_=C] (3,0) -- (1.5,0);
    \end{circuitikz}
  }\qquad
  \subfloat[]{
    \begin{circuitikz}[american voltages,label/.style={red}]
      % The two lines below - tf, why? Fix.
      \draw (0.3,1.9) node {\tiny{+}};
      \draw (0.3,1.15) node {\tiny{-}};
      \draw (0,3) to [battery,l_=\(V_{s}\)] (0,0);
      % Now not able to put node on switch on left side either
      \draw (0,3) to [normal open switch] (3,3) to [C,l_=C] (3,0) -- (0,0);
    \end{circuitikz}
  }
  \caption{Circuits}
\end{figure}

\subsection* {DATA}
The capacitance of both capacitors was measured and recorded in Table 1. The resistance in the first part was taken as 1 M\(\Omega\). The resistance for part 2 was measured as 3.85 k\(\Omega\).
\begin{table}
  \centering
  \begin{tabular}{| l | r | r |}
    \hline\hline
    Capacitor & Capacitance (F) & Average \\
    \hline
    \multirow{3}{*}{Large} & \(4.58\cdot{10}^{-5}\) & \\
    & \(4.58\cdot{10}^{-5}\) & \(4.58\cdot{10}^{-5}\) \\
    & \(4.58\cdot{10}^{-5}\) & \\
    \hline
    \multirow{3}{*}{Small} & \(3.62\cdot{10}^{-8}\) & \\
    & \(3.61\cdot{10}^{-8}\) & \(3.62\cdot{10}^{-8}\) \\
    & \(3.62\cdot{10}^{-8}\) & \\
    \hline\hline
  \end{tabular}
  \caption{Capacitance of the capacitors}
\end{table}

\begin{table}
  \centering
  \begin{tabular}{| l | r | r | r |}
    \hline\hline
    Time (s) & Voltage across capacitor (V) \\
    \hline
    0 & 7.585 \\
    \hline
    10 & 7.364 \\
    \hline
    12 & 7.063 \\
    \hline
    20 & 5.985 \\
    \hline
    28 & 5.078 \\
    \hline
    30 & 4.874 \\
    \hline
    39 & 4.055 \\
    \hline
    40 & 3.973 \\
    \hline
    50 & 3.242 \\
    \hline
    54 & 2.987 \\
    \hline
    60 & 2.645 \\
    \hline
    70 & 2.162 \\
    \hline
    74 & 1.994 \\
    \hline
    80 & 1.764 \\
    \hline
    90 & 1.443 \\
    \hline
    100 & 1.18 \\
    \hline
    106 & 1.049 \\
    \hline\hline
  \end{tabular}
  \caption{Voltage across large capacitor}
\end{table}

\noindent
For the second part of the lab, the oscilloscope was used to obtain at what times the capacitor was charged/discharged to 63\% of its highest voltage. Below is the data collected while the capacitor was discharging:

\noindent
For the first part of the graph:

\textbf{\(\Delta{V}\):} 3.76 V

\textbf{\(\Delta{t}\):} 150 \(\mu\)s

\noindent
For the second part of the graph:

\textbf{\(\Delta{V}\):} 3.72 V

\textbf{\(\Delta{t}\):} 140 \(\mu\)s

\noindent
Data collected as the capacitor was charging:

\textbf{\(\Delta{V}\):} 2.32 V

\textbf{\(\Delta{t}\):} 130 \(\mu\)s

\noindent
For the second part of the graph:

\textbf{\(\Delta{V}\):} 3.76 V

\textbf{\(\Delta{t}\):} 130 \(\mu\)s

\noindent
The \(\Delta{t}\) represents the change in time from when the capacitor first starts charging to when it has reached 63\% of its "maximum" voltage, while the change in voltage is the change in voltage across the capacitor from when it starts charging to when it is at its "maximum" voltage. 63\% of the change in voltage was calculated then to obtain the change in time. Thus, the average time constant is 137.5 \(\mu\)s, or \(1.35\cdot{10}^{-4}\) s for when the voltage across the capacitor is at 63\%.

\noindent
For Step 5, when the frequency is changed to 4.0 Hz, the period and voltage decrease as expected. Recall the relationship between period and frequency:
\begin{equation*}
  \begin{split}
    T &= \dfrac{1}{f} \\
  \end{split}
\end{equation*}
If the frequency is set higher, the period should then decrease.

\subsection* {ANALYSIS OF DATA}
For the first part of the lab, the large capacitor was used. It was charged using the voltage supply and then the voltage was recorded across the capacitor as it discharged. Figure 2 displays the data, where the red line showed the best-fit line for the data, while the blue line showed the expected line the data would follow. 

\noindent
Additionally, one of the supplemental questions shows that Equation 1 can be written as:
\begin{equation*}
  \begin{split}
    \ln(V(t)) &= \ln(V_{o}) - \frac{t}{\tau} \\
  \end{split}
\end{equation*}
Thus, the above equation can be used to find an experimental value of \(\tau\) by graphing \(\ln(V(t))\) versus \(t\), where -\sfrac{1}{\(\tau\)} is the slope. The result is shown in Figure 3. Then, since the slope of the best-fit line is equal to \(2\cdot{10}^{-2}\), then \(\tau\) is equal to:
\begin{equation*}
  \begin{split}
    -\tfrac{1}{\tau} &= -2\cdot{10}^{-2} \\
    \tau &= 50 \\
  \end{split}
\end{equation*}
A quick error analysis shows that this result has an error percentage of about 3.09\%, which shows the experimental value is very close to the predicated value.

\noindent
Recall Equation 1, which used the initial capacitor voltage, the time passed, and \(\tau\), the total resistance and the capacitor's capacitance. \(\tau\) is then calculated as:
\begin{equation*}
  \begin{split}
    \tau &= RC \\
         &= (1000000\Omega)(4.85\cdot{10}^{-5}) \\
         & = 48.5 \\
  \end{split}
\end{equation*}

\noindent
Then the equation for this specific circuit (Circuit a in Figure 1) is:
\begin{equation*}
  \begin{split}
    V_{a}(t) &= 7.585{e}^{-\tfrac{t}{48.5}} \\
  \end{split}
\end{equation*}
This equation is graphed in Figure 2 in blue. 

\pgfplotstableread{
X Y
0 7.585
10 7.364
12 7.063
20 5.985
28 5.078
30 4.874
39 4.055
40 3.973
50 3.242
54 2.987
60 2.645
70 2.162
74 1.994
80 1.764
90 1.443
100 1.182
106 1.049
}\largeVoltage

\begin {figure}                                                             
  \centering
  \begin{tikzpicture}
%    \draw[color=blue] plot (\x,7.585\e^{-\x/48.5});
    \begin{axis}[
      title = {Voltage of a Capacitor Vs. Time Passed},
      xlabel = {Time (s)},
      ylabel = {Voltage (V)},
      legend pos = north east,
      ]
      \addplot [only marks, mark = *] table {\largeVoltage};
      \addplot [thick, red] table[
        y={create col/linear regression={y=Y}}
      ]
      {\largeVoltage};
      \addplot[domain=0:106,blue] {7.585*exp(-0.0205394797*x)};
      \addlegendentry{Data}
      \addlegendentry{\((\pgfmathprintnumber{\pgfplotstableregressiona})x\pgfmathprintnumber[print sign]{\pgfplotstableregressionb}\)};
      \addlegendentry{\(7.585\cdot{e}^{-\tfrac{x}{48.5}}\)};
    \end{axis}
  \end{tikzpicture}
  \caption {}
\end {figure}

\pgfplotstableread{
X Y
0 2.026172613
10 1.996603263
12 1.95486989
28 1.789256339
30 1.583914955
39 1.399950688
40 1.379521477
50 1.176190423
54 1.094269539
60 0.9726710653
70 0.7710337192
74 0.6901426715
80 0.5675839576
90 0.3667242798
100 0.167207919
106 0.04783732941
}\second

\begin {figure}                                                             
  \centering
  \begin{tikzpicture}
%    \draw[color=blue] plot (\x,7.585\e^{-\x/48.5});
    \begin{axis}[
      title = {Natural Logarithm of the Voltage Across a Capacitor and Time Passed},
      xlabel = {Time (s)},
      ylabel = {Natural Log of Voltage (V)},
      legend pos = north east,
      ]
      \addplot [only marks, mark = *] table {\second};
      \addplot [thick, red] table[
        y={create col/linear regression={y=Y}}
      ]
      {\second};
      \addlegendentry{Data}
      \addlegendentry{\((\pgfmathprintnumber{\pgfplotstableregressiona})x\pgfmathprintnumber[print sign]{\pgfplotstableregressionb}\)};
      \addlegendentry{\(7.585\cdot{e}^{-\tfrac{x}{48.5}}\)};
    \end{axis}
  \end{tikzpicture}
  \caption {}
\end {figure}

\noindent
For the second part of the lab, the theoretical value of \(\tau\) was calculated as:
\begin{equation*}
  \begin{split}
    \tau &= RC \\
         &= (3850\Omega)(3.62\cdot{10}^{-8}) \\
         &= 0.00013937 \\
  \end{split}
\end{equation*}
The time constant from the lab was recorded with an average of \(1.35\cdot{10}^{-4}\), which gives an error percentage of about 1.34 \%.

\subsection* {DISCUSSION OF RESULTS}
For both procedures, the error percentage was very small (from 1 to 3\%), showing high consistency with the charging and discharging equations. Possible sources of error include neglecting to take into account the resistance of the large capacitor in the first part of the lab (discussed below) and minimal sources of voltage drop across the circuits (such as in the wires, for instance).

Recall that \(\tau\) is the product of the total resistance and the capacitance of the capacitor being charged. In the first part of the lab, the resistance of the large capacitor was not recorded. The resistance of the small capacitor was recorded as \(3.85\Omega\). Thus, if this value were used, then the \(\tau\) would be:
\begin{equation*}
  \begin{split}
    \tau &= RC \\
         &= (3850 \Omega + 1000000\Omega)(4.85\cdot{10}^{-5}\text{F}) \\
         &= 48.686725 \\
         &\approx 48.7 \\
  \end{split}
\end{equation*}
Notice that 48.7 is not far off from 48.5 as calculated previously, so while not recording the resistance of the capacitor was certainly a mistake, there was only a 0.412\% change from 48.5.

\subsection* {FURTHER STUDY}
Further implementation of this lab should be sure to include as much of the total resistance in the circuit into the calculations as possible. Additionally, it would be interesting to observe this time constant changing when other capacitors are added, and at different parts in the circuit.

\subsection* {SUPPLEMENTARY QUESTIONS}
\begin{enumerate}
  \item{Equation 2 can be written as \(\ln(V(t)) = \ln(V_{o})-\frac{t}{\tau}\). This means that if we plot \(\ln(V(t))\) versus \(t\), the slope will correspond to \(-\frac{1}{\tau}\). Find the natural logarithm of \(V(t)\) from your data and plot \(\ln(V(t))\) versus \(t\).}

  The graph is shown in Figure 3.
  \item{Find the slope of the best-fit line and thus obtain the experimentally measured value of the time constant. Compare the time constant obtained from this slope with the predicted value of \(\tau\) using \(\tau=RC\) and the values of \(R_{i}\) and \(C\) obtained previously.}

  The slope is shown in Figure 3, as well as discussed in the Analysis of Data section. An error percentage is included.
  \item{Compare your measured value with the product of \(RC\) obtained from the individual values of \(R\) and \(C\) measured earlier and equation \(\tau=RC\).}
  
  The theoretical and experimental values for \(\tau\) are discussed in sections Analysis of Data and Discussion of Results.  
  \item{Using Equation 2, show the mathematical reasoning behind why the time constant, \(\tau\), represents a 63\% decrease in the initial voltage for a discharging capacitor.}

  When time is equal to \(\tau\), then the discharging equation becomes:
  \begin{equation*}
    \begin{split}
      V_{c}(t) &= V_{0}e^{-1} \\
      V_{c}(t) &= \dfrac{V_{0}}{e} \\
      \dfrac{V_{c}(t)}{V_{0}} &= \dfrac{1}{e} \\
      \dfrac{V_{c}(t)}{V_{0}} &= 0.367879441 \\
    \end{split}
  \end{equation*}
  Thus, when time is equal to \(\tau\), the voltage across the capacitor at that point has discharged \((1-0.37)\cdot{100\%} = 63\%\).
  For the charging equation:
  \begin{equation*}
    \begin{split}
      V(t) &= V_{0}\left(1-e^{-1}\right) \\
      V(t) &= V_{0}-\dfrac{V_{0}}{e} \\
      \dfrac{V(t)}{V_{0}} &= 1-\dfrac{1}{e} \\
      \dfrac{V(t)}{V_{0}} &= 0.632120559 \\
    \end{split}
  \end{equation*}
  Again, showing that when time is equal to \(\tau\), the capacitor has charged to 63\% of its maximum voltage.
\end{enumerate}

\subsection* {REFERENCES}
Illinois Institute of Technology. (n.d.). Experiment 7: RC Circuits. PDF. Chicago.
\end {document}
