\documentclass [12pt, letterpaper, twoside] {article}
\usepackage[utf8]{inputenc}
\usepackage [left=1.0in, right=1.0in, top=1.0in, bottom=1.0in] {geometry}
% For updated time
\usepackage {datetime}
% For drawing pictures
\usepackage {tikz}
% For equations
\usepackage {amsmath}
% To make tables
\usepackage {tabu}
% For multiple rows in table slot
\usepackage {multirow}
\usepackage {verbatim}
% To add captions
\usepackage {caption}
\usepackage {float}
% To make graphs
\usepackage {pgfplots}
% To make scatter plots
\usepackage{pgfplotstable}

\usetikzlibrary {shapes.geometric, arrows, angles}

\tikzstyle {pink1circle0} = [circle, minimum size=0.5cm, text centered, draw=black, fill=pink1]
\tikzstyle {arrow} = [thick, ->, >=stealth]
\renewcommand {\labelitemiv}{$\triangle$}

\raggedbottom
\begin {document}
\begin {titlepage}
\begin {center}
College of Science: Physics Department \\
\vspace {0.1cm}
Illinois Institute of Technology \\
\vspace {0.1cm}
General Physics II: Electromagnetism (PHYS 221-01) \\
\vspace* {\fill}
\begingroup
\Large
\textbf {Capacitors}
\vspace {0.35cm}

\normalsize
Lab 4
\vspace {1.5cm}
\endgroup
\vspace* {\fill}
\end {center}

\vspace*{\fill}
\begin {flushright}
\footnotesize
Emily Pang, Lavanya Roy (lab partner) \\
Date of experiment: 19 Feb 2020 \\
Due date: 26 Feb 2020 \\
Lab section L06 \\
TA: Will Limestall \\
Updated \usdate\today~(\currenttime)
\end {flushright}
\end {titlepage}

\subsection* {Part 1 Questions}
Step 1:
The capacitance for the three capacitors was as follows:

\begin{table}[h!]
  \centering
  \begin{tabular}{| c | r |}
    \hline\hline
    Capacitor & Capacitance (F) \\
    \hline
    \multirow {3}{*}{\(C_{1}\)} & 3.62 \(\times{10}^{-8}\) \\
    & 3.63 \(\times{10}^{-8}\) \\
    & 3.62 \(\times{10}^{-8}\) \\
    \hline
    Average & 3.623333333 \(\times{10}^{-8}\) \\
    \hline
    \multirow {3}{*}{\(C_{2}\)} & 2.13 \(\times{10}^{-8}\) \\
    & 2.13 \(\times{10}^{-8}\) \\
    & 2.13 \(\times{10}^{-8}\) \\
    \hline
    Average & 2.13 \(\times{10}^{-8}\) \\
    \hline
    \multirow {3}{*}{\(C_{3}\)} & 8.1 \(\times{10}^{-9}\) \\
    & 8.1 \(\times{10}^{-9}\) \\
    & 8.1 \(\times{10}^{-9}\) \\
    \hline
    Average & 8.1 \(\times{10}^{-9}\) \\
    \hline\hline
  \end{tabular}
  \caption{Capacitor capacitance values}
\end{table}

\noindent
Step 2:
The effective (total) capacitance for the capacitors in series was recorded as:
\begin{table}[h!]
  \centering
  \begin{tabular}{| c | r |}
    \hline\hline
    Setup & \(C_{\text{total}}\) (F) \\
    \hline
    \multirow {3}{*}{Series} & 4.9 \(\times{10}^{-9}\) \\
    & 4.9 \(\times{10}^{-9}\) \\
    & 4.9 \(\times{10}^{-9}\) \\
    \hline
    Average & 4.9 \(\times{10}^{-9}\) \\
    \hline\hline
  \end{tabular}
  \caption{Total capacitance for the capacitors in series}
\end{table}

\noindent
The effective (total) capacitance for the capacitors in parallel was recorded as:
\begin{table}[h!]
  \centering
  \begin{tabular}{| c | r |}
    \hline\hline
    Setup & \(C_{\text{total}}\) (F) \\
    \hline
    \multirow {3}{*}{Parallel} & 6.57 \(\times{10}^{-8}\) \\
    & 6.56 \(\times{10}^{-8}\) \\
    & 6.57 \(\times{10}^{-8}\) \\
    \hline
    Average & 6.566666667 \(\times{10}^{-8}\) \\
    \hline\hline
  \end{tabular}
  \caption{Total capacitance for the capacitors in series}
\end{table}

Comparing the average values in Tables 1 and 2, it is shown that the capacitance in the parallel circuit was higher than in the series circuit. These values are corroborated with Equations . For instance, for the series circuit, \(C_{\text{total}}\) is:
\begin{equation*}
  \begin{split}
    C_{\text{total}} &= \dfrac{1}{\tfrac{1}{3.62\times10^{-8}C} + \tfrac{1}{2.13\times10^{-8}C} + \tfrac{1}{8.1\times10^{-9}C}} \\
    C_{\text{total}} &= 5.0497538\times10^{-9}C \\
  \end{split}
\end{equation*}
The parallel circuit was calculated as:
\begin{equation*}
  \begin{split}
    C_{\text{parallel}} &= 3.62\times10^{-8} + 2.13\times10^{-8} + 8.1\times10^{-9} \\
    C_{\text{parallel}} &= 6.56\times10^{-8} \\
  \end{split}
\end{equation*}

\noindent
Thus, it is shown that the capacitance is higher in a parallel rather than series circuit using equations and the experiment data.

\subsection* {Part 2a Questions}
Step 1:

\begin{table}[h!]
  \centering
  \begin{tabular}{| c | r |}
    \hline\hline
    Trial & Diameter of plate (m) \\
    \hline
    1 & 0.1985 \\
    \hline
    2 & 0.2011 \\
    \hline
    3 & 0.2005 \\
    \hline
    Average & 0.200033333 \\
    \hline\hline
  \end{tabular}
\end{table}

\begin{table}[h!]
  \centering
  \begin{tabular}{| c | r | r | r | r | r | r |}
    \hline\hline
    Distance (m) & 0.01 & 0.03 & 0.05 & 0.07 & 0.09 & 0.11 \\
    \hline
    \multirow {3}{*}{Capacitance (F)} & 3.94\(\times10^{-11}\) & 1.98\(\times10^{-11}\) & 1.58\(\times10^{-11}\) & 1.40\(\times10^{-11}\) & 1.31\(\times10^{-11}\) & 1.26\(\times10^{-11}\) \\
    & 3.95\(\times10^{-11}\) & 1.98\(\times10^{-11}\) & 1.58\(\times10^{-11}\) & 1.40\(\times10^{-11}\) & 1.32\(\times10^{-11}\) & 1.25\(\times10^{-11}\) \\
    & 3.93\(\times10^{-11}\) & 1.97\(\times10^{-11}\) & 1.58\(\times10^{-11}\) & 1.40\(\times10^{-11}\) & 1.32\(\times10^{-11}\) & 1.25\(\times10^{-11}\) \\
    \hline
    Average & 3.94\(\times10^{-11}\) & 1.98\(\times10^{-11}\) & 1.58\(\times10^{-11}\) & 1.40\(\times10^{-11}\) & 1.32\(\times10^{-11}\) & 1.25\(\times10^{-11}\) \\ % %6666667 (LDR) % % %6666667 (LDR) %3333333 (LDR)
    \hline\hline
  \end{tabular}
  \caption{Distance between plates and the resulting capacitance using air as a dialectric}
\end{table}

Step 2:
\begin{table}[h!]
  \centering
  \begin{tabular}{| c | r | r | r | r | r |}
    \hline\hline
    Distance (m) & 1 sheet (0.0001 m) & 10 sheets (0.00010 m) \\
    \hline
    \multirow {3}{*}{Capacitance (F) Before} & 3.98\(\times10^{-8}\) & 3.98\(\times10^{-8}\) \\
    & 3.98\(\times10^{-8}\) & 3.98\(\times10^{-8}\) \\
    & 3.95\(\times10^{-8}\) & 3.96\(\times10^{-8}\) \\
    \hline
    Average & 3.97\(\times10^{-8}\) & 3.97\(\times10^{-8}\) \\
    \hline
    \multirow {3}{*}{Capacitance (F) After} & 4.00\(\times10^{-8}\) & 4.21\(\times10^{-8}\) \\
    & 4.00\(\times10^{-8}\) & 4.21\(\times10^{-8}\) \\
    & 3.98\(\times10^{-8}\) & 4.19\(\times10^{-8}\) \\
    \hline
    Average & 3.993333333\(\times10^{-8}\) & 4.203333333\(\times10^{-8}\) \\
    \hline\hline
  \end{tabular}
  \caption{Distance between plates and the resulting capacitance using paper as a dialectric}
\end{table}

\begin{table}[h!]
  \centering
  \begin{tabular}{| c | r | r | r | r | r |}
    \hline\hline
    Distance (m) & 1 sheet & 3 sheets \\
    \hline
    \multirow {3}{*}{Capacitance (F) Before} & 4.00\(\times10^{-8}\) & 3.96\(\times10^{-8}\) \\
    & 3.94\(\times10^{-8}\) & 3.95\(\times10^{-8}\) \\
    & 3.95\(\times10^{-8}\) & 3.95\(\times10^{-8}\) \\
    \hline
    Average & 3.963333333\(\times10^{-8}\) & 3.953333333\(\times10^{-8}\) \\
    \hline
    \multirow {3}{*}{Capacitance (F) After} & 4.02\(\times10^{-8}\) & 4.05\(\times10^{-8}\) \\
    & 3.97\(\times10^{-8}\) & 4.02\(\times10^{-8}\) \\
    & 3.98\(\times10^{-8}\) & 4.02\(\times10^{-8}\) \\
    \hline
    Average & 3.99\(\times10^{-8}\) & 4.03\(\times10^{-8}\) \\
    \hline\hline
  \end{tabular}
  \caption{Distance between plates and the resulting capacitance using cellulose as a dielectric}
\end{table}

\begin{table}
  \centering
  \begin{tabular}{| l | r |}
    \hline\hline
    Distance (m) & Capacitance (F) \\
    \hline
    0.0030 & 2.19\(\times10^{-10}\) \\
    \hline
    0.0030 & 2.15\(\times10^{-10}\) \\
    \hline
    0.0030 & 2.13\(\times10^{-10}\) \\
    \hline\hline
  \end{tabular}
  \caption{Distance between plates and the resulting capacitance using vinyl as a dielectric}
\end{table}

\begin{table}
  \centering
  \begin{tabular}{| l | r |}
    \hline\hline
    Distance (m) & Capacitance (F) \\
    \hline
    0.0030 & 2.19\(\times10^{-10}\) \\
    \hline
    0.0030 & 2.15\(\times10^{-10}\) \\
    \hline
    0.0030 & 2.13\(\times10^{-10}\) \\
    \hline\hline
  \end{tabular}
  \caption{Distance between plates and the resulting capacitance using nylon as a dielectric}
\end{table}

\begin{table}
  \centering
  \begin{tabular}{| l | r |}
    \hline\hline
    Distance (m) & Capacitance (F) \\
    \hline
    0.0080 & 8.54\(\times10^{-11}\) \\
    \hline
    0.0081 & 8.59\(\times10^{-11}\) \\
    \hline
    0.0080 & 8.56\(\times10^{-11}\) \\  
    \hline\hline
  \end{tabular}
  \caption{Distance between plates and capacitance (with vinyl on left and nylon on right)}
\end{table}

\begin{table}
  \centering
  \begin{tabular}{| l | r |}
    \hline\hline
    Distance (m) & Capacitance (F) \\
    \hline
    0.0075 & 9.94\(\times10^{-11}\) \\
    \hline
    0.0075 & 1.000\(\times10^{-10}\) \\
    \hline
    0.0075 & 9.98\(\times10^{-11}\) \\  
    \hline\hline
  \end{tabular}
  \caption{Distance between plates and capacitance (with nylon on left and vinyl on right)}
\end{table}

\subsection* {Main Lab Questions}
\begin{enumerate}
  \item Calculate the capacitance using Equation 1 for the various separation distances of task 1 (for part 2a). How do these values compare with the data taken with the capacimeter?  Don’t forget to take into account the dielectric constant for air.

  The equation for calculating capacitance is:
  \begin{equation*}
    \begin{split}
      C &= \kappa\dfrac{\epsilon_{0}A}{d} \\
    \end{split}
  \end{equation*}
  The dielectric constant for air is:
  \begin{equation*}
    \begin{split}
      \kappa &= 1.00054 \\
    \end{split} 
  \end{equation*}
  Thus, the expected values for the capacitance using the above equations  and the average area of the plates are shown in Table 11.
  \begin{table}
    \centering
    \begin{tabular}{| l | r |}
      \hline\hline
      Distance (m) & Calculated Capacitance (F) \\
      \hline
      0.01 & 2.77016086\(\times10^{-11}\) \\
      \hline
      0.03 & 9.23386952\(\times10^{-12}\) \\
      \hline
      0.05 & 5.54032171\(\times10^{-12}\) \\
      \hline
      0.07 & 3.95737265\(\times10^{-12}\) \\
      \hline
      0.09 & 3.07795651\(\times10^{-12}\) \\
      \hline
      0.11 & 2.51832805\(\times10^{-12}\) \\
      \hline\hline
    \end{tabular}
    \caption{Distance between plates and the calculated capacitance}
  \end{table}
  A quick error analysis in Table 12 shows that the experimental values are .
  \begin{table}
    \centering
    \begin{tabular}{| l | r |}
      \hline\hline
      Distance (m) & Error percentage \\
      \hline
      0.01 & \\
      \hline
      0.03 & \\
      \hline
      0.05 & \\
      \hline
      0.07 & \\
      \hline
      0.09 & \\
      \hline
      0.11 & \\
      \hline\hline
    \end{tabular}
    \caption{Error percentages when using air as a dielectric}
  \end{table}
   
  \item  Calculate the dielectric constant for the paper used.  How do your values compare with the listed value from the table above?
  \item Calculate the dielectric constant for the transparency sheets.  Does the capacitance of the parallel plates depend on the thickness of the dielectric?
  \item Show that the capacitance for the setup in Figure 4 is given by
  \begin{equation}
    \begin{split}
      C &= \dfrac{2\epsilon_{0}A}{d}\dfrac{\kappa_{1}\kappa_{2}}{\kappa_{1}+\kappa_{2}} \\
    \end{split}
\end{equation}
  \item Compare your experimental results of task 4 with Equation 4.
  \item The dielectrics you insert between the parallel plates may have excess charge.  What happens when you charge the plates, if this turns out to be true?
  \item For Part 2b, create a plot of V vs d.  What does the slope represent?
  \item How much charge is found on the plates in Part 2b?
\end{enumerate}

\pgfplotstableread {
X Y
0.0038 1
0.0072 2.01
0.0105 3.00
0.0145 4.00
0.0177 5.00
}\circuita

\pgfplotstableread {
X Y
0.0035 1
0.0070 2.01
0.0105 3.02
0.0140 3.99
0.0175 4.99
}\circuitb

\begin {figure}
  \centering
  \begin{tikzpicture}
    \begin{axis}[
      title = {Circuit A Voltage Vs. Amperage},
      xlabel = {Voltage (V)},
      ylabel = {Amperage (A)},
      legend pos=north west,
      ]
      \addplot [only marks, mark = *] table {\circuita};
      \addplot [thick, red] table[
        y={create col/linear regression={y=Y}}
      ]
      {\circuita};
      \addlegendentry{Data}
      \addlegendentry{\((\pgfmathprintnumber{\pgfplotstableregressiona})x\pgfmathprintnumber[print sign]{\pgfplotstableregressionb}\)}
    \end{axis}
  \end{tikzpicture}
  \caption {}
\end {figure}

\begin {figure}
  \centering
  \begin{tikzpicture}
    \begin{axis}[
      title = {Circuit B Voltage Vs. Amperage},
      xlabel = {Voltage (V)},
      ylabel = {Amperage (A)},
      legend pos = north west,
      ]
      \addplot [only marks, mark = *] table {\circuitb};
      \addplot [thick, red] table[
        y={create col/linear regression={y=Y}}
      ]
      {\circuitb};
      \addlegendentry{Data}
      \addlegendentry{\((\pgfmathprintnumber{\pgfplotstableregressiona})x\pgfmathprintnumber[print sign]{\pgfplotstableregressionb}\)}
    \end{axis}
  \end{tikzpicture}
  \caption {}
\end {figure}

\subsection* {REFERENCES}
Amer Chaaban, M. (n.d.). Voltage Drop. Retrieved February 12, 2020, from https://www.e-education.psu.edu/ae868/node/967 \\\\
Gladding, G., Selen, M. A., \& Stelzer, T. (2012). Electricity and Magnetism. New York: W.H. Freeman. \\\\
Illinois Institute of Technology. (n.d.). Experiment 1: Coulomb's Law. PDF. Chicago.
\end {document}
