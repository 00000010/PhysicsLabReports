\documentclass [12pt, letterpaper, twoside] {article}
\usepackage[utf8]{inputenc}
\usepackage [left=1.0in, right=1.0in, top=1.0in, bottom=1.0in] {geometry}
% For updated time
\usepackage {datetime}
% For drawing pictures
\usepackage {tikz}
% For equations
\usepackage {amsmath}
% To make tables
\usepackage {tabu}
% For multiple rows in table slot
\usepackage {multirow}
\usepackage {verbatim}
% To add captions
\usepackage {caption}
\usepackage {float}
% To make graphs
\usepackage {pgfplots}
% To make scatter plots
\usepackage{pgfplotstable}
% To make tables side by side
\usepackage{subfig}

\usetikzlibrary {shapes.geometric, arrows, angles}

\tikzstyle {pink1circle0} = [circle, minimum size=0.5cm, text centered, draw=black, fill=pink1]
\tikzstyle {arrow} = [thick, ->, >=stealth]
\renewcommand {\labelitemiv}{$\triangle$}

\raggedbottom
\begin {document}
\begin {titlepage}
\begin {center}
College of Science: Physics Department \\
\vspace {0.1cm}
Illinois Institute of Technology \\
\vspace {0.1cm}
General Physics II: Electromagnetism (PHYS 221-01) \\
\vspace* {\fill}
\begingroup
\Large
\textbf {Complex Circuits}
\vspace {0.35cm}

\normalsize
Lab 6
\vspace {1.5cm}
\endgroup
\vspace* {\fill}
\end {center}

\vspace*{\fill}
\begin {flushright}
\footnotesize
Emily Pang, Lavanya Roy (lab partner) \\
Date of experiment: 4 Mar 2020 \\
Due date: 11 Mar 2020 \\
Lab section L06 \\
TA: Will Limestall \\
Updated \usdate\today~(\currenttime)
\end {flushright}
\end {titlepage}

\subsection* {Configuration 1}
\begin {figure}
  \centering
  \subfloat[Case A] {
    \begin {tikzpicture}
      \draw[style=ultra thick] (0,0) rectangle (8.75,5.125);

      \filldraw[fill=red,style=thick] (0.50,4.625) circle (0.375cm);
      \filldraw[fill=white,style=thick] (0.50,4.625) circle (0.20cm);
      \draw (0.50,4.625) node {1};

      \filldraw[fill=blue!70!red!70!,style=thick] (4.375,4.625) circle (0.375cm);
      \filldraw[fill=white,style=thick] (4.375,4.625) circle (0.20cm);
      \draw (4.375,4.625) node {2};

      \filldraw[fill=white,style=thick] (8.25,4.625) circle (0.375cm);
      \filldraw[fill=white,style=thick] (8.25,4.625) circle (0.20cm);
      \draw (8.25,4.625) node {3};

      \filldraw[fill=yellow,style=thick] (0.50,0.50) circle (0.375cm);
      \filldraw[fill=white,style=thick] (0.50,0.50) circle (0.20cm);
      \draw (0.50,0.50) node {4};

      \filldraw[fill=blue,style=thick] (4.375,0.50) circle (0.375cm);
      \filldraw[fill=white,style=thick] (4.375,0.50) circle (0.20cm);
      \draw (4.375,0.50) node {5};

      \filldraw[fill=green,style=thick] (8.25,0.5) circle (0.375cm);
      \filldraw[fill=white,style=thick] (8.25,0.5) circle (0.20cm);
      \draw (8.25,0.5) node {6};

      % Connections
      % 2 to 5
      \draw[style=thick] (4.375,4.25) -- (4.375,3.2);
      \draw[style=thick,decoration = {zigzag,segment length = 3mm, amplitude = 1mm},decorate] (4.375,3.2) -- (4.375,1.925) node[right] {\(R_{1}\)};
      \draw[style=thick] (4.375,1.925) -- (4.375,0.875);

      % 4 to 5
      \draw[style=thick] (0.875,0.50) -- (1.925,0.50);
      \draw[style=thick,decoration = {zigzag,segment length = 3mm, amplitude = 1mm},decorate] (1.925,0.50) -- (2.95,0.50) node[above] {\(R_{2}\)};
      \draw[style=thick] (2.95,0.50) -- (4,0.50);
      
      % 2 to 6

      \draw[style=thick] (4.630,4.350) -- (5.865,3.017);
      \draw[style=thick,decoration = {zigzag,segment length = 3mm, amplitude = 1mm}, decorate] (5.865,3.017) -- (6.731,2.081) node[right] {\(R_{3}\)};
      \draw[style=thick] (6.731,2.081) -- (7.995,0.775);

    \end {tikzpicture}
  }\qquad
  \subfloat[Case B (correct orientation)] {
    \begin {tikzpicture}
      \draw[style=ultra thick] (0,0) rectangle (8.75,5.125);

      \filldraw[fill=red,style=thick] (0.50,4.625) circle (0.375cm);
      \filldraw[fill=white,style=thick] (0.50,4.625) circle (0.20cm);
      \draw (0.50,4.625) node {1};

      \filldraw[fill=blue!70!red!70!,style=thick] (4.375,4.625) circle (0.375cm);
      \filldraw[fill=white,style=thick] (4.375,4.625) circle (0.20cm);
      \draw (4.375,4.625) node {2};

      \filldraw[fill=white,style=thick] (8.25,4.625) circle (0.375cm);
      \filldraw[fill=white,style=thick] (8.25,4.625) circle (0.20cm);
      \draw (8.25,4.625) node {3};

      \filldraw[fill=yellow,style=thick] (0.50,0.50) circle (0.375cm);
      \filldraw[fill=white,style=thick] (0.50,0.50) circle (0.20cm);
      \draw (0.50,0.50) node {4};

      \filldraw[fill=blue,style=thick] (4.375,0.50) circle (0.375cm);
      \filldraw[fill=white,style=thick] (4.375,0.50) circle (0.20cm);
      \draw (4.375,0.50) node {5};

      \filldraw[fill=green,style=thick] (8.25,0.5) circle (0.375cm);
      \filldraw[fill=white,style=thick] (8.25,0.5) circle (0.20cm);
      \draw (8.25,0.5) node {6};

      % Connections
      % 2 to 5
      \draw[style=thick] (4.375,4.25) -- (4.375,3.2);
      \draw[style=thick,decoration = {zigzag,segment length = 3mm, amplitude = 1mm},decorate] (4.375,3.2) -- (4.375,1.925) node[right] {\(R_{1}\)};
      \draw[style=thick] (4.375,1.925) -- (4.375,0.875);

      % 4 to 5
      \draw[style=thick] (0.875,0.50) -- (1.925,0.50);
      \draw[style=thick,decoration = {zigzag,segment length = 3mm, amplitude = 1mm},decorate] (1.925,0.50) -- (2.95,0.50) node[above] {\(R_{2}\)};
      \draw[style=thick] (2.95,0.50) -- (4,0.50);
    
      % 5 to 6
      \draw[style=thick] (4.75,0.50) -- (5.795,0.50);
      \draw[style=thick,decoration = {zigzag,segment length = 3mm, amplitude = 1mm},decorate] (5.795,0.50) -- (7.07,0.50) node[above] {\(R_{3}\)};
      \draw[style=thick] (7.07,0.50) -- (7.875,0.50);
    \end {tikzpicture}
  } 
  \caption {Possible Box A6 Configurations}
\end {figure}

\begin{table}
  \centering
  \begin{tabular}{| l | r | r | r | r | r | r |}
    \hline\hline
    Connection & 1 & 2 & 3 & 4 & 5 & 6 \\
    \hline
    1 & & 0 & 0 & 0 & 0 & 0 \\
    \hline
    2 & &  & 0 & 0.020 & 0.045 & 0.019 \\
    \hline
    3 & & & & 0 & 0 & 0 \\
    \hline
    4 & & & & & 0.037 & 0.017 \\
    \hline
    5 & & & & & & 0.032 \\
    \hline\hline
  \end{tabular}
  \caption {Amperage at different connections (Box A6)}
\end{table}

Looking at Table 1, the different current values can be used to find the resistance when the voltage is known to be 10.058 V.
\begin{equation}
  \begin{split}
    I = \dfrac{V}{R} \\
  \end{split}
\end{equation}
Then for each of the connections the resistances are:

\vspace{0.5cm}
2-4: 502.9 \(\Omega\)

2-5: 223.5111111 \(\Omega\) (Error \%: 1.595959591)

2-6: 529.3684211 \(\Omega\)

4-5: 271.8378378 \(\Omega\) (Error \%: 0.680680667)

4-6: 591.6470588 \(\Omega\)

5-6: 314.3125 \(\Omega\)
\vspace{0.5cm}

Since the two known resistors have values of 220\(\Omega\pm\) 5\% (\(R_{1}\)) and 270\(\Omega\pm\) 5\% (\(R_{2}\)), then the 2-5 connection must contain the 220\(\Omega\) resistor and the 4-5 connection must contain the 270\(\Omega\) resistor. The error percentages are listed below these.

The unknown resistor (\(R_{3}\)) must then reside between 2 and 6 or 5 and 6, because 223.51 \(\Omega\) and 271.84 \(\Omega\) roughly add up to 502.90 \(\Omega\) for between connection 2 and 4. Additionally, \(R_{3}\) cannot be between 2 and 6 (Case A in Figure 1) because then the value of \(R_{3}\) would be both 529.37 \(\Omega\) (by connection 2 to 6) and 90.80 \(\Omega\) (by connection 5 to 6). For Case B, the resistance of \(R_{3}\) would be 314.31 \(\Omega\) for connection 5 to 6 and 305.86 \(\Omega\) for connection 2 to 6 by adding resistances as they are in series.

\subsection* {Configuration 2}
Equation 1 can be used again for finding the total resistance for connections. The amperage for Box B6 is shown in Table 2. Then, for each of the connections the resistances are (where voltage equals 10.047 V):

\begin {table}
  \centering
  \begin {tabular}{| l | l | l | l | l | l | l |}
    \hline\hline
    Connection & 1 & 2 & 3 & 4 & 5 & 6 \\
    \hline
    1 & & 0 & 0 & 0 & 0 & 0 \\
    \hline
    2 & & & 0 & 0 & 0 & 0 \\
    \hline
    3 & & & & 0.067 & 0 & 0 \\
    \hline
    4 & & & & & 0 & 0 \\
    \hline
    5 & & & & & & 0 \\
    \hline\hline
  \end {tabular}
  \caption {Amperage at different connections (Box B6)}
\end {table}

\vspace{0.5cm}
3-4: 149.9552239 \(\Omega\)
\vspace{0.5cm}

Notice that 149.96 \(\Omega\) is much smaller than \(R_{1}\) or \(R_{2}\), meaning that the resistors cannot be in series, as then \(R_{3}\) would have to be negative. Resistance for resistors in parallel is defined as:
\begin{equation}
  \begin{split}
    R_{\text{total}} &= \dfrac{1}{\tfrac{1}{R_{1}} + \tfrac{1}{R_{2}} + \tfrac{1}{R_{3}}} \\
  \end{split}
\end{equation}
Then, solving for \(R_{3}\):
\begin{equation*}
  \begin{split}
    R_{\text{total}} &= \dfrac{1}{\tfrac{R_{2}R_{3}}{R_{1}R_{2}R_{3}} + \tfrac{R_{1}R_{3}}{R_{1}R_{2}R_{3}} + \tfrac{R_{1}R_{2}}{R_{1}R_{2}R_{3}}} \\
    R_{\text{total}} &= \dfrac{1}{\tfrac{R_{2}R_{3} + R_{1}R_{3} + R_{1}R_{2}}{R_{1}R_{2}R_{3}}} \\
    R_{\text{total}} &= \dfrac{R_{1}R_{2}R_{3}}{R_{2}R_{3} + R_{1}R_{3} + R_{1}R_{2}} \\
    R_{\text{total}}(R_{2}R_{3} + R_{1}R_{3} + R_{1}R_{2}) &= R_{1}R_{2}R_{3} \\
    R_{\text{total}}R_{2}R_{3} + R_{\text{total}}R_{1}R_{3} + R_{\text{total}}R_{1}R_{2} &= R_{1}R_{2}R_{3} \\
    R_{\text{total}}R_{2}R_{3} + R_{\text{total}}R_{1}R_{3} - R_{1}R_{2}R_{3} &= -R_{\text{total}}R_{1}R_{2} \\
    R_{3}(R_{\text{total}}R_{2} + R_{\text{total}}R_{1} - R_{1}R_{2}) &= -R_{\text{total}}R_{1}R_{2} \\
    R_{3} &= \dfrac{-R_{\text{total}}R_{1}R_{2}}{R_{\text{total}}R_{2} + R_{\text{total}}R_{1} -R_{1}R_{2}} \\
  \end{split}
\end{equation*}

However, evaluating \(R_{3}\) with \(R_{1}\) with 220 \(\Omega\), \(R_{2}\) with 270 \(\Omega\) and \(R_{\text{total}}\) as 149.96 \(\Omega\) gives a value of -623 \(\Omega\). Thus, the only alternatives are if two of the resistors are in parallel. These two configurations are shown in Figure 2. In order to find the range of \(R_{3}\), the following formula is used for Case A:
\begin{equation*}
  \begin{split}
    R_{\text{total}} &= R_{1} + \dfrac{1}{\tfrac{1}{R_{2}} + \tfrac{1}{R_{3}}} \\
  \end{split}
\end{equation*}
If \(R_{3}\) is solved for:
\begin{equation*}
  \begin{split}
    R_{\text{total}} &= R_{1} + \dfrac{1}{\tfrac{R_{3}}{R_{2}R_{3}} + \tfrac{R_{2}}{R_{2}R_{3}}} \\
    R_{\text{total}} &= R_{1} + \dfrac{1}{\tfrac{R_{2} + R_{3}}{R_{2}R_{3}}} \\
    R_{\text{total}} &= R_{1} + \dfrac{R_{2}R_{3}}{R_{2} + R_{3}} \\
    R_{\text{total}} - R_{1} &= \dfrac{R_{2}R_{3}}{R_{2} + R_{3}} \\
    (R_{\text{total}} - R_{1})(R_{2} + R_{3} ) &= R_{2}R_{3} \\
    R_{\text{total}}R_{2} + R_{\text{total}}R_{3} - R_{1}R_{2} - R_{1}R_{3} &= R_{2}R_{3} \\
    R_{\text{total}}R_{3} -R_{1}R_{3} - R_{2}R_{3} &= -R_{\text{total}}R_{2} + R_{1}R_{2} \\
    R_{3}(R_{\text{total}} - R_{1} - R_{2}) &=  -R_{\text{total}}R_{2} + R_{1}R_{2} \\
    R_{3} &= \dfrac{-R_{\text{total}}R_{2} + R_{1}R_{2}}{R_{\text{total}} - R_{1} - R_{2}} \\
  \end{split}
\end{equation*}

\noindent
For the above solution, \(R_{3}\) is equal to -55.61646841 \(\Omega\). Since \(R_{3}\) cannot have a negative value, this configuration is not possible.

\noindent
For Case B, the equation is:
\begin{equation*}
  \begin{split}
    R_{\text{total}} &= R_{2} + \dfrac{1}{\tfrac{1}{R_{1}} + \tfrac{1}{R_3}} \\
  \end{split}
\end{equation*}
If \(R_{3}\) is solved for:
\begin{equation*}
  \begin{split}
    R_{3} &= \dfrac{-R_{\text{total}}R_{1} + R_{1}R_{2}}{R_{\text{total}} - R_{1} - R_{2}} \\
  \end{split}
\end{equation*}

\noindent
For the above solution, \(R_{3}\) is equal to -77.66580344 \(\Omega\). Since \(R_{3}\) cannot have a negative value, this configuration is not possible.

\noindent
For Case C, the equation is:
\begin{equation*}
  \begin{split}
    R_{\text{total}} &= R_{3} + \dfrac{1}{\tfrac{1}{R_{1}} + \tfrac{1}{R_{2}}} \\
  \end{split}
\end{equation*}
If \(R_{3}\) is solved for:
\begin{equation*}
  \begin{split}
    R_{3} &= R_{\text{total}} - \dfrac{1}{\tfrac{1}{R_{1}} + \tfrac{1}{R_{2}}} \\
  \end{split}
\end{equation*}
For the above solution, \(R_{3}\) is equal to 28.7307341 \(\Omega\). Since this is the only value that is feasible for \(R_{3}\), then this configuration is most likely.

\begin {figure}
  \centering
    \subfloat[Case A] {
      \begin {tikzpicture}
        \draw[style=ultra thick] (0,0) rectangle (8.75,5.125);

        \filldraw[fill=red,style=thick] (0.50,4.625) circle (0.375cm);
        \filldraw[fill=white,style=thick] (0.50,4.625) circle (0.20cm);
        \draw (0.50,4.625) node {1};

        \filldraw[fill=blue!70!red!70!,style=thick] (4.375,4.625) circle (0.375cm);
        \filldraw[fill=white,style=thick] (4.375,4.625) circle (0.20cm);
        \draw (4.375,4.625) node {2};

        \filldraw[fill=white,style=thick] (8.25,4.625) circle (0.375cm);
        \filldraw[fill=white,style=thick] (8.25,4.625) circle (0.20cm);
        \draw (8.25,4.625) node {3};

        \filldraw[fill=yellow,style=thick] (0.50,0.50) circle (0.375cm);
        \filldraw[fill=white,style=thick] (0.50,0.50) circle (0.20cm);
        \draw (0.50,0.50) node {4};

        \filldraw[fill=blue,style=thick] (4.375,0.50) circle (0.375cm);
        \filldraw[fill=white,style=thick] (4.375,0.50) circle (0.20cm);
        \draw (4.375,0.50) node {5};

        \filldraw[fill=green,style=thick] (8.25,0.5) circle (0.375cm);
        \filldraw[fill=white,style=thick] (8.25,0.5) circle (0.20cm);
        \draw (8.25,0.5) node {6};

        % Connections 
        \draw [style=thick] (0.831030035,0.676193406) -- (3.2375,2.5626);
        \draw [style=thick] (3.2375,2.5626) -- (3.2375,3.0626) -- (3.7375,3.0626);
        \draw [style=thick] (3.2375,2.5626) -- (3.2375,2.0626) -- (3.7375,2.0626);
        \draw[style=thick,decoration = {zigzag,segment length = 3mm, amplitude = 1mm},decorate] (3.7375,3.0626) -- (5.0125,3.0626) node[above] {\(R_{2}\)};
        \draw[style=thick,decoration = {zigzag,segment length = 3mm, amplitude = 1mm},decorate] (3.7375,2.0626) -- (5.0125,2.0626) node[below] {\(R_{3}\)};
        \draw[style=thick] (5.0125,3.0626) -- (5.5125,3.0626) -- (5.5125,2.5626);
        \draw[style=thick] (5.0125,2.0626) -- (5.5125,2.0626) -- (5.5125,2.5626);
        \draw[style=thick] (5.5125,2.5626) -- (6.213812569,3.11229329);
        \draw[style=thick,decoration = {zigzag,segment length = 3mm, amplitude = 1mm},decorate] (6.213812569,3.11229329) -- (7.217297835,3.898831478) node[above] {\(R_{1}\)};
        \draw[style=thick] (7.217297835,3.898831478) -- (7.918969965, 4.448806594);
      \end {tikzpicture}
    }
    
    \subfloat[Case B] {
      \begin {tikzpicture}
        \draw[style=ultra thick] (0,0) rectangle (8.75,5.125);

        \filldraw[fill=red,style=thick] (0.50,4.625) circle (0.375cm);
        \filldraw[fill=white,style=thick] (0.50,4.625) circle (0.20cm);
        \draw (0.50,4.625) node {1};

        \filldraw[fill=blue!70!red!70!,style=thick] (4.375,4.625) circle (0.375cm);
        \filldraw[fill=white,style=thick] (4.375,4.625) circle (0.20cm);
        \draw (4.375,4.625) node {2};

        \filldraw[fill=white,style=thick] (8.25,4.625) circle (0.375cm);
        \filldraw[fill=white,style=thick] (8.25,4.625) circle (0.20cm);
        \draw (8.25,4.625) node {3};

        \filldraw[fill=yellow,style=thick] (0.50,0.50) circle (0.375cm);
        \filldraw[fill=white,style=thick] (0.50,0.50) circle (0.20cm);
        \draw (0.50,0.50) node {4};

        \filldraw[fill=blue,style=thick] (4.375,0.50) circle (0.375cm);
        \filldraw[fill=white,style=thick] (4.375,0.50) circle (0.20cm);
        \draw (4.375,0.50) node {5};

        \filldraw[fill=green,style=thick] (8.25,0.5) circle (0.375cm);
        \filldraw[fill=white,style=thick] (8.25,0.5) circle (0.20cm);
        \draw (8.25,0.5) node {6};

        % Connections 
        \draw [style=thick] (0.831030035,0.676193406) -- (3.2375,2.5626);
        \draw [style=thick] (3.2375,2.5626) -- (3.2375,3.0626) -- (3.7375,3.0626);
        \draw [style=thick] (3.2375,2.5626) -- (3.2375,2.0626) -- (3.7375,2.0626);
        \draw[style=thick,decoration = {zigzag,segment length = 3mm, amplitude = 1mm},decorate] (3.7375,3.0626) -- (5.0125,3.0626) node[above] {\(R_{1}\)};
        \draw[style=thick,decoration = {zigzag,segment length = 3mm, amplitude = 1mm},decorate] (3.7375,2.0626) -- (5.0125,2.0626) node[below] {\(R_{3}\)};
        \draw[style=thick] (5.0125,3.0626) -- (5.5125,3.0626) -- (5.5125,2.5626);
        \draw[style=thick] (5.0125,2.0626) -- (5.5125,2.0626) -- (5.5125,2.5626);
        \draw[style=thick] (5.5125,2.5626) -- (6.213812569,3.11229329);
        \draw[style=thick,decoration = {zigzag,segment length = 3mm, amplitude = 1mm},decorate] (6.213812569,3.11229329) -- (7.217297835,3.898831478) node[above] {\(R_{2}\)};
        \draw[style=thick] (7.217297835,3.898831478) -- (7.918969965, 4.448806594);
      \end {tikzpicture}
    }\qquad
    \subfloat[Case C] {
      \begin {tikzpicture}
        \draw[style=ultra thick] (0,0) rectangle (8.75,5.125);

        \filldraw[fill=red,style=thick] (0.50,4.625) circle (0.375cm);
        \filldraw[fill=white,style=thick] (0.50,4.625) circle (0.20cm);
        \draw (0.50,4.625) node {1};

        \filldraw[fill=blue!70!red!70!,style=thick] (4.375,4.625) circle (0.375cm);
        \filldraw[fill=white,style=thick] (4.375,4.625) circle (0.20cm);
        \draw (4.375,4.625) node {2};

        \filldraw[fill=white,style=thick] (8.25,4.625) circle (0.375cm);
        \filldraw[fill=white,style=thick] (8.25,4.625) circle (0.20cm);
        \draw (8.25,4.625) node {3};

        \filldraw[fill=yellow,style=thick] (0.50,0.50) circle (0.375cm);
        \filldraw[fill=white,style=thick] (0.50,0.50) circle (0.20cm);
        \draw (0.50,0.50) node {4};

        \filldraw[fill=blue,style=thick] (4.375,0.50) circle (0.375cm);
        \filldraw[fill=white,style=thick] (4.375,0.50) circle (0.20cm);
        \draw (4.375,0.50) node {5};

        \filldraw[fill=green,style=thick] (8.25,0.5) circle (0.375cm);
        \filldraw[fill=white,style=thick] (8.25,0.5) circle (0.20cm);
        \draw (8.25,0.5) node {6};

        % Connections 
        \draw [style=thick] (0.831030035,0.676193406) -- (3.2375,2.5626);
        \draw [style=thick] (3.2375,2.5626) -- (3.2375,3.0626) -- (3.7375,3.0626);
        \draw [style=thick] (3.2375,2.5626) -- (3.2375,2.0626) -- (3.7375,2.0626);
        \draw[style=thick,decoration = {zigzag,segment length = 3mm, amplitude = 1mm},decorate] (3.7375,3.0626) -- (5.0125,3.0626) node[above] {\(R_{1}\)};
        \draw[style=thick,decoration = {zigzag,segment length = 3mm, amplitude = 1mm},decorate] (3.7375,2.0626) -- (5.0125,2.0626) node[below] {\(R_{2}\)};
        \draw[style=thick] (5.0125,3.0626) -- (5.5125,3.0626) -- (5.5125,2.5626);
        \draw[style=thick] (5.0125,2.0626) -- (5.5125,2.0626) -- (5.5125,2.5626);
        \draw[style=thick] (5.5125,2.5626) -- (6.213812569,3.11229329);
        \draw[style=thick,decoration = {zigzag,segment length = 3mm, amplitude = 1mm},decorate] (6.213812569,3.11229329) -- (7.217297835,3.898831478) node[above] {\(R_{3}\)};
        \draw[style=thick] (7.217297835,3.898831478) -- (7.918969965, 4.448806594);
      \end {tikzpicture}
    }
  \caption {Possible Box B6 Configurations}
\end {figure}

\thebibliography{3}
  \bibitem{labManual}
  Illinois Institute of Technology. (n.d.). Experiment 6: Complex Circuits. PDF. Chicago.
\end {document}
