\documentclass [12pt, letterpaper, twoside] {article}
\usepackage[utf8]{inputenc}
\usepackage [left=1.0in, right=1.0in, top=1.0in, bottom=1.0in] {geometry}
% For updated time
\usepackage {datetime}
% For drawing pictures
\usepackage {tikz}
% For equations
\usepackage {amsmath}
% To make tables
\usepackage {tabu}
% For multiple rows in table slot
\usepackage {multirow}
\usepackage {verbatim}
% To add captions
\usepackage {caption}
\usepackage {float}
% To make graphs
\usepackage {pgfplots}
% To make scatter plots
\usepackage{pgfplotstable}
% To make tables side by side
\usepackage{subfig}

\usetikzlibrary {shapes.geometric, arrows, angles}

\tikzstyle {pink1circle0} = [circle, minimum size=0.5cm, text centered, draw=black, fill=pink1]
\tikzstyle {arrow} = [thick, ->, >=stealth]
\renewcommand {\labelitemiv}{$\triangle$}

\raggedbottom
\begin {document}
\begin {titlepage}
\begin {center}
College of Science: Physics Department \\
\vspace {0.1cm}
Illinois Institute of Technology \\
\vspace {0.1cm}
General Physics II: Electromagnetism (PHYS 221-01) \\
\vspace* {\fill}
\begingroup
\Large
\textbf {Complex Circuits}
\vspace {0.35cm}

\normalsize
Lab 6
\vspace {1.5cm}
\endgroup
\vspace* {\fill}
\end {center}

\vspace*{\fill}
\begin {flushright}
\footnotesize
Emily Pang, Lavanya Roy (lab partner) \\
Date of experiment: 4 Mar 2020 \\
Due date: 11 Mar 2020 \\
Lab section L06 \\
TA: Will Limestall \\
Updated \usdate\today~(\currenttime)
\end {flushright}
\end {titlepage}

\subsection* {Configuration 1}
\begin {figure}
  \centering
  \subfloat {
    \begin {tikzpicture}
      \draw[style=ultra thick] (0,0) rectangle (8.75,5.125);

      \filldraw[fill=red,style=thick] (0.50,4.625) circle (0.375cm);
      \filldraw[fill=white,style=thick] (0.50,4.625) circle (0.20cm);
      \draw (0.50,4.625) node {1};

      \filldraw[fill=blue!70!red!70!,style=thick] (4.375,4.625) circle (0.375cm);
      \filldraw[fill=white,style=thick] (4.375,4.625) circle (0.20cm);
      \draw (4.375,4.625) node {2};

      \filldraw[fill=white,style=thick] (8.25,4.625) circle (0.375cm);
      \filldraw[fill=white,style=thick] (8.25,4.625) circle (0.20cm);
      \draw (8.25,4.625) node {3};

      \filldraw[fill=yellow,style=thick] (0.50,0.50) circle (0.375cm);
      \filldraw[fill=white,style=thick] (0.50,0.50) circle (0.20cm);
      \draw (0.50,0.50) node {4};

      \filldraw[fill=blue,style=thick] (4.375,0.50) circle (0.375cm);
      \filldraw[fill=white,style=thick] (4.375,0.50) circle (0.20cm);
      \draw (4.375,0.50) node {5};

      \filldraw[fill=green,style=thick] (8.25,0.5) circle (0.375cm);
      \filldraw[fill=white,style=thick] (8.25,0.5) circle (0.20cm);
      \draw (8.25,0.5) node {6};

      % Connections
      % 2 to 5
      \draw[style=thick] (4.375,4.25) -- (4.375,3.2);
      \draw[style=thick,decoration = {zigzag,segment length = 3mm, amplitude = 1mm},decorate] (4.375,3.2) -- (4.375,1.925);
      \draw[style=thick] (4.375,1.925) -- (4.375,0.875);

      % 4 to 5
      \draw[style=thick] (0.875,0.50) -- (1.925,0.50);
      \draw[style=thick,decoration = {zigzag,segment length = 3mm, amplitude = 1mm},decorate] (1.925,0.50) -- (2.95,0.50);
      \draw[style=thick] (2.95,0.50) -- (4,0.50);
      
      % 2 to 6

      \draw[style=thick] (4.630,4.350) -- (5.865,3.017);
      \draw[style=thick,decoration = {zigzag,segment length = 3mm, amplitude = 1mm}, decorate] (5.865,3.017) -- (6.731,2.081);
      \draw[style=thick] (6.731,2.081) -- (7.995,0.775);

    \end {tikzpicture}
  }
  \subfloat {
    \begin {tikzpicture}
      \draw[style=ultra thick] (0,0) rectangle (8.75,5.125);

      \filldraw[fill=red,style=thick] (0.50,4.625) circle (0.375cm);
      \filldraw[fill=white,style=thick] (0.50,4.625) circle (0.20cm);
      \draw (0.50,4.625) node {1};

      \filldraw[fill=blue!70!red!70!,style=thick] (4.375,4.625) circle (0.375cm);
      \filldraw[fill=white,style=thick] (4.375,4.625) circle (0.20cm);
      \draw (4.375,4.625) node {2};

      \filldraw[fill=white,style=thick] (8.25,4.625) circle (0.375cm);
      \filldraw[fill=white,style=thick] (8.25,4.625) circle (0.20cm);
      \draw (8.25,4.625) node {3};

      \filldraw[fill=yellow,style=thick] (0.50,0.50) circle (0.375cm);
      \filldraw[fill=white,style=thick] (0.50,0.50) circle (0.20cm);
      \draw (0.50,0.50) node {4};

      \filldraw[fill=blue,style=thick] (4.375,0.50) circle (0.375cm);
      \filldraw[fill=white,style=thick] (4.375,0.50) circle (0.20cm);
      \draw (4.375,0.50) node {5};

      \filldraw[fill=green,style=thick] (8.25,0.5) circle (0.375cm);
      \filldraw[fill=white,style=thick] (8.25,0.5) circle (0.20cm);
      \draw (8.25,0.5) node {6};

      % Connections
      % 2 to 5
      \draw[style=thick] (4.375,4.25) -- (4.375,3.2);
      \draw[style=thick,decoration = {zigzag,segment length = 3mm, amplitude = 1mm},decorate] (4.375,3.2) -- (4.375,1.925) node[right] {\(R_{1}\)};
      \draw[style=thick] (4.375,1.925) -- (4.375,0.875);

      % 4 to 5
      \draw[style=thick] (0.875,0.50) -- (1.925,0.50);
      \draw[style=thick,decoration = {zigzag,segment length = 3mm, amplitude = 1mm},decorate] (1.925,0.50) -- (2.95,0.50) node[above] {\(R_{2}\)};
      \draw[style=thick] (2.95,0.50) -- (4,0.50);
    
      % 5 to 6
      \draw[style=thick] (4.75,0.50) -- (5.795,0.50);
      \draw[style=thick,decoration = {zigzag,segment length = 3mm, amplitude = 1mm},decorate] (5.795,0.50) -- (7.07,0.50) node[above] {\(R_{3}\)};
      \draw[style=thick] (7.07,0.50) -- (7.875,0.50);
    \end {tikzpicture}
  } 
  \caption {Possible Box A6 Configurations}
\end {figure}

\begin{table}
  \centering
  \begin{tabular}{| l | r | r | r | r | r | r |}
    \hline\hline
    Connection & 1 & 2 & 3 & 4 & 5 & 6 \\
    \hline
    1 & & 0 & 0 & 0 & 0 & 0 \\
    \hline
    2 & &  & 0 & 0.020 & 0.045 & 0.019 \\
    \hline
    3 & & & & 0 & 0 & 0 \\
    \hline
    4 & & & & & 0.037 & 0.017 \\
    \hline
    5 & & & & & & 0.032 \\
    \hline\hline
  \end{tabular}
  \caption {Amperage at different connections}
\end{table}

Looking at Table 1, the different current values can be used to find the resistance when the voltage is known to be 10.058 V.
\begin{equation}
  \begin{split}
    I = \dfrac{V}{R} \\
  \end{split}
\end{equation}
Then, for each of the connections the resistances are:

2-4: 502.9 \(\Omega\)

2-5: 223.5111111 \(\Omega\)

2-6: 529.3684211 \(\Omega\)

4-5: 271.8378378 \(\Omega\)

4-6: 591.6470588 \(\Omega\)

5-6: 314.3125 \(\Omega\)

Since the two known resistors have values of 220\(\Omega\pm\) 5\% (\(R_{1}\)) and 270\(\Omega\pm\) 5\% (\(R_{2}\)), then the 2-5 connection must contain the 220\(\Omega\) resistor and the 4-5 connection must contain the 270\(\Omega\) resistor.

The unknown resistor (\(R_{3}\)) must then reside between 2 and 6 or 5 and 6, because 223.51 \(\Omega\) and 271.84 \(\Omega\) roughly add up to 502.90 \(\Omega\) for between connection 2 and 4.

\thebibliography{3}
  \bibitem{labManual}
  Illinois Institute of Technology. (n.d.). Experiment 6: Complex Circuits. PDF. Chicago.
\end {document}
