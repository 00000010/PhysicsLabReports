\documentclass [12pt, letterpaper, twoside] {article}
\usepackage[utf8]{inputenc}
\usepackage [left=1.0in, right=1.0in, top=1.0in, bottom=1.0in] {geometry}
% For updated time
\usepackage {datetime}
\usepackage {tikz}
% For equations
\usepackage {amsmath}
% To make tables
\usepackage {tabu}
% For multiple rows in table slot
\usepackage {multirow}
\usepackage {verbatim}
% To add captions
\usepackage {caption}
\usepackage {float}
% To make graphs
\usepackage {pgfplots}
% To make scatter plots
\usepackage {pgfplotstable}
% To make electric field diagrams
%\usepackage {pstricks}
% To make electric field diagrams
%\usepackage {pst-electricfield}

\usetikzlibrary{patterns, petri,positioning}

\raggedbottom
\pgfplotsset{compat=1.16}
\begin {document}
\begin {titlepage}
\begin {center}
Department of Biological, Chemical, and Physical Science\\
\vspace {0.1cm}
Illinois Institute of Technology\\
\vspace {0.1cm}
General Physics II: Electromagnetism (PHYS 221-01)\\
\vspace* {\fill}
\begingroup
\Large
\textbf {Electric Fields and Electric Potential}
\vspace {0.35cm}

\normalsize
Lab 3
\vspace {1.5cm}
\endgroup
\vspace* {\fill}
\end {center}

\vspace*{\fill}
\begin {flushright}
\footnotesize
Emily Pang, Lavanya Roy (lab partner) \\
Date of experiment: 12 Feb 2020 \\
Due date: 19 Feb 2020 \\
Lab section L06 \\
TA: Will Limestall \\
Updated \usdate\today~(\currenttime)
\end {flushright}
\end {titlepage}
\subsection* {STATEMENT OF OBJECTIVE}
The objective of this lab was to understand the relationship between field lines and equipotential lines using concepts of electricity, including Coulomb's Law and electric potential.

\subsection* {THEORY}
Voltage is defined as the electric potential, which is defined as
\begin{equation}
  \begin{split}
    V - V_{0} = -\int_{P_{0}}^{P}E\cos(\theta)ds \\
  \end{split}
\end{equation}
where the voltage difference is measured in Volts and is the summation of the electric field from the points which you are calculating to (i.e. \(P\) to \(P_{0}\)).\cite{labManual} When examining the voltage between two charges, these values can be mapped to create equipotential lines, or areas where the potential is the same. Electric field lines are drawn from charges to show the direction of the charge's force. Additionally, these lines are always perpendicular to equipotential lines.

\subsection* {EQUIPMENT}
  \noindent
  \begin {itemize}
    \itemsep0em
    \item {two sheets of black conductive paper}
    \item {conductive ink}
    \item {two metal push pins}
    \item {multimeter}
    \item {wires to connect the push pins and voltage supply}
    \item {voltage supply}
    \item {at least two pieces of grid paper}
    \item {cork board}
  \end {itemize}

\subsection* {PROCEDURE}
First the conductive ink was drawn onto the conductive paper for two charges. For the first experiment, the charges were two dipolar points, while the second was any shape of two charges. The conductive paper was then placed onto the cork board and secured with the push pins so that the pins were in directly contact with the ink. Figure 1 shows the configuration for the first experiment.

\begin {figure}
  \centering
  \begin {tikzpicture}
    \draw[pattern=north west lines, pattern color=black] (0,0) rectangle (7.5,0.5);
    \draw[pattern=north west lines, pattern color=black] (0,6.5) rectangle (7.5,7);
    \draw[pattern=dots, pattern color=black] (0,0.5) rectangle (7.5,6.5);
    (2,3.5) circle (3,4);
  \end {tikzpicture}   
  \caption {First experiment setup}
\end {figure}

\subsection* {DATA}

\subsection* {ANALYSIS OF DATA}

\subsection* {DISCUSSION OF RESULTS}

\subsection* {FURTHER STUDY}

\subsection* {SUPPLEMENTARY QUESTIONS}
\begin{enumerate}
  \item How does the potential vary with distance on your plots?

The greater the distance between a point on the conductive paper and a charge in a circuit, the greater the potential, or voltage. This is supported by Equation 
  \item Calculate the electric field strength at a few places with different characteristics using Eq. 2. Do your results agree with the idea that electric field line density is proportional electric field strength? Sketch neatly and clearly an appropriate amount of lines on your grid paper. You may want to use a different color.
  \item Consider the following situation: An object with charge \(q_{0} = 1.5\mu\) C and mass 0.7 g starts from rest at the +6V equipotential line. Calculate its change in potential energy and speed when it reaches the +2V line.
\end {enumerate}

\subsection* {REFERENCES}
\thebibliography{3}
  \bibitem{labManual}
  Illinois Institute of Technology. (n.d.). Experiment 3: Electric Fields and Electric Potential. PDF. Chicago.
\end {document}
