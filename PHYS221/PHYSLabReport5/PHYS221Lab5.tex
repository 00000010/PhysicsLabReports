% Document notes:
% Improvements:
%   - In Figure 3, make the total voltage table go directly underneath the current table and to the right of the resistor voltage table.
%   - There may be a bug with the circuitikz package where it should be able to put plus/minus signs with batteries (last heard of in 2014)
%   - Cannot draw voltage across resistors in parallel circuit? May be because they are vertical.
%   - Fixed all the sloppy typesetting (i.e. warnings like underbox 10000)
\documentclass [12pt, letterpaper, twoside] {article}
\usepackage[utf8]{inputenc}
\usepackage [left=1.0in, right=1.0in, top=1.0in, bottom=1.0in] {geometry}
% For updated time
\usepackage {datetime}
% For drawing pictures
\usepackage {tikz}
% For equations
\usepackage {amsmath}
% To make tables
\usepackage {tabu}
% For multiple rows in table slot
\usepackage {multirow}
\usepackage {verbatim}
% To add captions
\usepackage {caption}
\usepackage {float}
% To make graphs
\usepackage {pgfplots}
% To make scatter plots
\usepackage{pgfplotstable}
% To make pretty theorems
\usepackage[english]{babel}
% To make circuit drawings
\usepackage{circuitikz}
% To make tables side by side
\usepackage{subfig}

\usetikzlibrary {shapes.geometric, arrows, angles}
\newtheorem{theorem}{Theorem}

\raggedbottom
\pgfplotsset{compat=1.16}
\begin {document}
\begin {titlepage}
\begin {center}
College of Science: Physics Department \\
\vspace {0.1cm}
Illinois Institute of Technology\\
\vspace {0.1cm}
General Physics II: Electromagnetism (PHYS 221-01)\\
\vspace* {\fill}
\begingroup
\Large
\textbf {Simple Resistor Circuits}
\vspace {0.35cm}

\normalsize
Lab 5 
\vspace {1.5cm}
\endgroup
\vspace* {\fill}
\end {center}

\vspace*{\fill}
\begin {flushright}
\footnotesize
Emily Pang, Lavanya Roy (lab partner) \\
Date of experiment: 26 Feb 2020 \\
Due date: 4 Mar 2020 \\
Lab section L06 \\
TA: Will Limestall \\
Updated \usdate\today~(\currenttime)
\end {flushright}
\end {titlepage}
\subsection* {STATEMENT OF OBJECTIVE}
The objective of this lab was to examine Ohm's Law, its connection with resistors, and how resistance changes in series, parallel, or both types of circuits.

\subsection* {THEORY}
\noindent
A circuit can be described by Ohm's Law, which is represented by the following:
\begin{equation}
  \begin{split}
    V &= IR \\
  \end{split}
\end{equation}
where \(V\) is the voltage measured in Volts (V), \(I\) is the current measured in Amps (A), and \(R\) is the resistance measured in Ohms (\(\Omega\)). Two other concepts will help describe circuits, namely Kirchhoff's First Law and Kirchhoff's Second Law.
\begin{theorem}[Kirchhoff's First Law]
  The current entering a junction is the same as the current leaving a junction.
\end{theorem}
\begin{theorem}[Kirchhoff's Second Law]
  The sum of all the voltage drops around a closed circuit is equal to zero.
\end{theorem}
Furthermore, resistance in a circuit is described differently depending on whether it is in series or parallel. If the resistors are in series, their resistances are added together.
\begin{equation}
  \begin{split}
    R_{\text{total}} &= R_{1} + R_{2} + R_{3} + \cdots \\
  \end{split}
\end{equation}
If the resistances are in parallel, the total resistance is the inverse of the addition of the inverses of all of the resistances.
\begin{equation}
  \begin{split}
    R_{\text{total}} &= \dfrac{1}{\tfrac{1}{R_{1}} + \tfrac{1}{R_{2}} + \tfrac{1}{R_{3}} + \cdots} \\
  \end{split}
\end{equation}
Current in a circuit will result in different values depending on the placement of the ammeter. The current in resistors in series is equal, while current in parallel splits at junctions by Kirchhoff's First Law. 

\subsection* {EQUIPMENT}
  \noindent
  \begin {itemize}
    \itemsep0em
    \item {four different resistors}
    \item {one multimeter}
    \item {one ammeter}
    \item {about 4 stripped wires}
    \item {power supply}
    \item {one breadboard}
    \item {wires to connect power supply and breadboard}
  \end {itemize}

\subsection* {PROCEDURE}
Each of the four resistors were first measured by the multimeter. Then all but the fourth resistor were constructed into a series circuit, as shown in Figure 1, diagram a. By Ohm's Law (Equation 1), resistance can be calculated if both the voltage across the resistor and the current through the resistor are known. Thus, the multimeter was used to measure the voltage and the ammeter was used to measure the current for each resistor. Ohm's Law (Equation 1) can again be used for finding the total resistance experimentally. As for resistor 4, it was added to the series circuit and its value was used to obtain an experimental value of the current.

\begin{figure}
  \centering
  \subfloat[Series circuit]{
    \begin{circuitikz}[]
      \draw (0,0) -- (0,2)
      to[R=\(R_{1}\), v_=\(V_{1}\)] (2,2) -- (2,2)
      to[R=\(R_{1}\), v_=\(V_{2}\)] (3,2) -- (3,2)
      to[R=\(R_{3}\), v_=\(V_{3}\)] (5,2) -- (5,0) to[battery1, l=\(V\), i=\(I\)] (0,0);
    \end{circuitikz}
  }\qquad
  \subfloat[Parallel circuit]{
    \begin{circuitikz}[american]
      \draw (4.5,0) -- (0,0) to[R=\(R_{1}\)] (0,3) -- (4.5,3) to[battery1, l=\(V\), i=\(I\)] (4.5,0);
      \draw (1.5,0) to[R=\(R_{2}\)] (1.5,3);
      \draw (3,0) to[R=\(R_{3}\)] (3,3);
    \end{circuitikz}
  }\qquad
  \subfloat[Combined circuit]{
    \begin{circuitikz}[]
      \draw (4,4.5) to[battery1, l=\(V\), i=\(I\)] (4,0) -- (0,0);
      \draw (4,4.5) -- (0,4.5) -- (0,4.5) to[R=\(R_{1}\)] (0,2.5);
      \draw (0,2.5) -- (-0.75,2.5) to[R=\(R_{2}\)] (-0.75,0.5) -- (0.75,0.5);
      \draw (0,2.5) -- (0.75,2.5) to[R=\(R_{3}\)] (0.75,0.5);
      \draw (0,0.5) -- (0,0);
    \end{circuitikz}
  }
  \caption{Circuit configurations}
\end{figure}

The second part of the experiment examined resistors in parallel, and was constructed as in Figure 1, diagram b. The current through each resistor in the parallel circuit was measured, while the experimental total resistance was obtained using Ohm's Law (Equation 1) and the total measured current and voltage. Total resistance was then found again using Equation 3.

The last part of the experiment analyzed a combination of the series and parallel circuits and used the circuit in Figure 1, diagram c. In order to validate Kirchhoff's Laws, \(I_{1}\), \(I_{2}\), \(I_{3}\), \(V_{\text{total}}\), \(V_{1}\), \(V_{2}\), and \(V_{3}\) were measured and then compared to the expected values using Kirchhoff's Laws. 
    
\subsection* {DATA}
The resistances as measured by the multimeter and calculated for series and parallel circuits are shown in Table 1. The colors on each of the resistors are shown below: 

\vspace{0.5cm}
\indent
\textbf{\(R_{1}\):} Red, violet, brown, gold (\(27\times10^{1}\pm5\%\text{ }\Omega\))

\indent
\textbf{\(R_{2}\):} Green, blue, brown, gold (\(56\times10^{1}\pm5\%\text{ }\Omega\))

\indent
\textbf{\(R_{3}\):} Brown, brown, red, gold (\(11\times10^{2}\pm5\%\text{ }\Omega\))

\indent
\textbf{\(R_{4}\):} Orange, orange, red, silver (\(33\times10^{2}\pm10\%\text{ }\Omega\))
\vspace{0.5cm}

\begin{table}
  \centering
  \begin{tabular}{| l | r | c | c | c | c |}
    \hline\hline
    \multirow {3}{*}{Resistor} & & Calculated & \% & Calculated & \% \\
    & Resistance (\(\Omega\)) & Resistance, & Error & Resistance, & Error \\
    & & Series (\(\Omega\)) & & Parallel (\(\Omega\)) & \\
    \hline
    \multirow {3}{*}{\(R_{1}\)} & 270 & & & & \\
    & 268 & & & & \\
    & 269 & & & & \\
    \hline
    Average & 269 & 303.96 & 13.00 & 289.03 & 7.45 \\ %97112 (LDR) %617517 (LDR) %10786 %6497621 (LDR)
    \hline
    \multirow {3}{*}{\(R_{2}\)} & 549 & & & & \\
    & 549 & & & & \\
    & 549 & & & & \\
    \hline
    Average & 549 & 616.11 & 12.22 & 585.56 & 6.66 \\ %73412 (LDR) %355942 %5556 (LDR) %857122 (LDR)
    \hline
    \multirow {3}{*}{\(R_{3}\)} & 1085 & & & & \\
    & 1085 & & & & \\
    & 1085 & & & & \\
    \hline
    Average & 1085 & 1220.13 & 12.45 & 1188.7218 & 9.56 \\ %4146 %476 %9612903 (LDR)
    \hline
    \multirow {3}{*}{\(R_{4}\)} & 3560 & & & & \\
    & 3560 & & & & \\
    & 3560 & & & & \\
    \hline
    Average & 3560 & 3916.98 & 10.03 & 3809.64 & 7.01 \\ %0393 %753913 (LDR) %8095 (LDR) %2306039
    \hline\hline
  \end{tabular}
  \caption{Resistances by multimeter, in series, and in parallel circuits}
\end{table}

\noindent
The voltage across each of the resistors and the total current are shown in Figure 2. For the series circuit with \(R_{4}\), the voltage (\(V_{4}\)) and total current is again measured, as shown in Figure 3. Next, data was collected for the current running through each resistor when all resistors were in a parallel circuit. The total voltage was measured again. These results are shown in Figure 4.
The data from procedure 2 of the experiment is presented in Figure 5.

\begin{figure}
  \centering
  \subfloat[Voltage across each resistor]{
    \begin{tabular}{| l | r |}
      \hline\hline
      Voltage across Resistor & Voltage (V) \\
      \hline
      \multirow {3}{*}{\(V_{1}\)} & 1.509 \\
      & 1.510 \\
      & 1.510 \\
      \hline
      Average & 1.510 \\ %666667 (LDR)
      \hline
      \multirow {3}{*}{\(V_{2}\)} & 3.06 \\
      & 3.06 \\
      & 3.06 \\
      \hline
      Average & 3.06 \\
      \hline
      \multirow {3}{*}{\(V_{3}\)} & 6.06 \\
      & 6.06 \\
      & 6.06 \\
      \hline
      Average & 6.06 \\
      \hline\hline
    \end{tabular}
  }\qquad
  \subfloat[Total current]{
    \begin{tabular}{| l | r |}
      \hline\hline
      Current & Current (A) \\
      \hline
      \multirow {3}{*}{\(I_{\text{total}}\)} & 0.0050 \\
      & 0.0049 \\
      & 0.0050 \\
      \hline
      Average & 0.0050 \\ %66667 (LDR)
      \hline\hline
    \end{tabular}
  }\qquad
  \subfloat[Total voltage]{
    \begin{tabular}{| l | r |}
      \hline\hline
      Voltage & Voltage (V) \\
      \hline
      \multirow {3}{*}{\(V_{\text{total}}\)} & 10.63 \\
      & 10.63 \\
      & 10.63 \\
      \hline
      Average & 10.63 \\
      \hline\hline
    \end{tabular}
  }
  \caption{Resistors in series without \(R_{4}\)}
\end{figure}

\begin{figure}
  \centering
  \subfloat[Voltage]{
    \begin{tabular}{| l | r |}
      \hline\hline
      Voltage across Resistor & Voltage (V) \\
      \hline
      \multirow {3}{*}{\(V_{4}\)} & 6.92 \\
      & 6.92 \\
      & 6.92 \\
      \hline
      Average & 6.92 \\
      \hline\hline
    \end{tabular}
  }\qquad
  \subfloat[Current]{
    \begin{tabular}{| l | r |}
      \hline\hline
      Total current & Current (A) \\
      \hline
      \multirow {3}{*}{\(I_{\text{total}}\)} & 0.0018 \\
      & 0.0017 \\
      & 0.0018 \\
      \hline
      Average & 0.001766667 \\
      \hline\hline
    \end{tabular}
  }
  \caption{Resistors in series with \(R_{4}\)}
\end{figure}
    
\begin{figure}
  \centering
  \subfloat[Current through each resistor]{
    \begin{tabular}{| l | r |}
      \hline\hline
      Current through resistor & Current (A) \\
      \hline
      \multirow {3}{*}{\(I_{1}\)} & 0.0365 \\
      & 0.0365 \\
      & 0.0364 \\
      \hline
      Average & 0.0365 \\ %6666667 (LDR)
      \hline
      \multirow {3}{*}{\(I_{2}\)} & 0.0180 \\
      & 0.0180 \\
      & 0.0180 \\
      \hline
      Average & 0.0180 \\
      \hline
      \multirow {3}{*}{\(I_{3}\)} & 0.0089 \\
      & 0.0089 \\
      & 0.0088 \\
      \hline
      Average & 0.0089 \\ %666667 (LDR)
      \hline
      \multirow {3}{*}{\(I_{4}\)} & 0.0027 \\
      & 0.0028 \\
      & 0.0028 \\
      \hline
      Average & 0.0028 \\ %66667 (LDR)
      \hline\hline
    \end{tabular}
  }\qquad
  \subfloat[Total voltage]{
    \begin{tabular}{| l | r |}
      \hline\hline
      Voltage & Voltage (V) \\
      \hline
      \multirow {3}{*}{\(V_{\text{total}}\)} & 10.54 \\
      & 10.54 \\
      & 10.54 \\
      \hline
      Average & 10.54 \\
      \hline
    \end{tabular}
  }
  \caption{Resistors in parallel}
\end{figure}

\begin{figure}
  \centering
  \subfloat[Current through resistors]{
    \begin{tabular}{| l | r |}
      \hline\hline
      Current through resistor & Current (A) \\
      \hline
      \multirow {3}{*}{\(I_{1}\)} & 0.0160 \\
      & 0.0160 \\
      & 0.0160 \\
      \hline
      Average & 0.0160 \\
      \hline
      \multirow {3}{*}{\(I_{2}\)} & 0.0103 \\
      & 0.0104 \\
      & 0.0103 \\
      \hline
      Average & 0.0103 \\ %33333
      \hline
      \multirow {3}{*}{\(I_{3}\)} & 0.0051 \\
      & 0.0051 \\
      & 0.0051 \\
      \hline
      Average & 0.0051 \\
      \hline\hline
    \end{tabular}
  }
  \subfloat[Voltage across resistors]{
    \begin{tabular}{| l | r |}
      \hline\hline
      Voltage across resistor & Voltage (V) \\
      \hline
      \multirow {3}{*}{\(V_{1}\)} & 4.49 \\
      & 4.49 \\
      & 4.49 \\
      \hline
      Average & 4.49 \\
      \hline
      \multirow {3}{*}{\(V_{2}\)} & 6.07 \\
      & 6.08 \\
      & 6.08 \\
      \hline
      Average & 6.08 \\ %6666667 (LDR)
      \hline
      \multirow {3}{*}{\(V_{3}\)} & 6.08 \\
      & 6.08 \\
      & 6.08 \\
      \hline
      Average & 6.08 \\
      \hline\hline
    \end{tabular}
  }\qquad
  \subfloat[Total voltage]{
    \begin{tabular}{| l | r |}
      \hline\hline
      Voltage & Voltage (V) \\
      \hline
      \multirow {3}{*}{\(V_{\text{total}}\)} & 10.58 \\
      & 10.58 \\
      & 10.58 \\
      \hline
      Average & 10.58 \\
      \hline\hline
    \end{tabular}
  }
  \caption{Resistors in series and parallel combination}
\end{figure}

\subsection* {ANALYSIS OF DATA}
In order to evaluate resistance in a series circuit, the experimental values obtained by measuring the voltage across each resistor and the total current were compared to the established resistance values using Equation 1. Since the current going through the resistors is identical in series circuits, each resistor value was found by the following:
\begin{equation*}
  \begin{split}
    R_{n} = \dfrac{V_{n}}{I_{\text{total}}} \\
  \end{split}
\end{equation*}
where \(R_{n}\) is the resistance and \(V_{n}\) is the voltage of some resistor, \(n\), in a series circuit. The results of the calculations are shown in Table 1. For the parallel circuit, the individual resistances were calculated using Ohm's Law again (Equation 1) and the fact that voltage across resistors in parallel is equal. The results of these calculations are shown in Table 1. Overall, when calculating the resistor values whether in a series or parallel circuit, the values were relatively close to each other, as witnessed in Table 1. The highest error percentage was around 13\%, which is a relatively modest difference. Using the values for the current through each resistor and the measured resistance using the multimeter, the result can be viewed pictorally in Figure 6, where the slope represents the voltage. However, the slope, 9.83, representing 9.83 V, had an error of about 7.53\% from the expected 10.63 V. %550843 (LDR) %9246945 (LDR) %5870179 (LDR)

\pgfplotstableread{
X Y
3.717472119 36.46666667
1.821493625 18.0
0.921658986 8.866667
0.280898876 2.766667
}\parallel

\begin {figure}
  \centering
  \begin{tikzpicture}
    \begin{axis}[
      title = {Current through each resistor and inverse of resistor resistance},
      xlabel = {Inverse of resistance (k\(\Omega\))},
      ylabel = {Current through resistor (kA)},
      legend pos = north west,
      ]
      \addplot [only marks, mark = *] table {\parallel};
      \addplot [thick, red] table[
        y={create col/linear regression={y=Y}}
      ]
      {\parallel};
      \addlegendentry{Data}
      \addlegendentry{\((\pgfmathprintnumber{\pgfplotstableregressiona})x\pgfmathprintnumber[print sign]{\pgfplotstableregressionb}\)}
    \end{axis}
  \end{tikzpicture}
  \caption {}
\end {figure}


For the total resistances in the series and parallel circuits, the theoretical values were caluclated using Equation 2 and 3, respectively. The experimental results for each circuit were measured by using Ohm's Law and Kirchhoff's Second Law for the series circuit, and Ohm's Law for the parallel circuit. For the series circuit, Kirchhoff's Second Law holds that:
\begin{equation}
  \begin{split}
    V_{\text{total}} &= V_{1} + V_{2} + V_{3} + \cdots + V_{n} \\
  \end{split}
\end{equation}
Then, Ohm's Law shows that:
\begin{equation}
  \begin{split}
    I_{\text{total}}R_{\text{total}} &= I_{1}R_{1} + I_{2}R_{2} + I_{3}R_{3} + \cdots + I_{n}R_{n} \\
    R_{\text{total}} &= \dfrac{I_{1}R_{1} + I_{2}R_{2} + I_{3}R_{3} + \cdots + I_{n}R_{n}}{I_{\text{total}}} \\
  \end{split}
\end{equation}
Since the current in a series circuit is equal through all the resistors, \(I_{\text{total}} = I_{1} = I_{2} = I_{3} = I_{n}\). The following is then given:
\begin{equation*}
  \begin{split}
    R_{\text{total}} &= R_{1} + R_{2} + R_{3} + \cdots + R_{n} \\
  \end{split}
\end{equation*}
Thus, the experimental total resistance is simply the experimental resistances found using Ohm's Law summed together. For the parallel circuit, the following was used:
\begin{equation*}
  \begin{split}
    R_{\text{total}} &= \dfrac{V_{\text{total}}}{I_{\text{total}}} \\
  \end{split}
\end{equation*}
since \(V_{\text{total}} = V_{1} = V_{2} = V_{3}\) in a parallel circuit when each resistor is in a separate branch of the circuit. The results for both calculations are shown in Table 2. Error percentages show that the results are consistent with the values obtained theoretically, validating Ohms's Law and the two equations for calculating resistances in series and parallel circuits.

\begin{table}
  \centering
  \begin{tabular}{| l | c | c | c | c | c | c |}
    \hline\hline
    \multirow {3}{*}{Resistor} & Theoretical & Experimental & \% & Theoretical & Experimental & \% \\
    & Resistance, & Resistance, & Error & Resistance & Resistance, & Error \\
    & Series (\(\Omega\)) & Series (\(\Omega\)) & & Parallel (\(\Omega\)) & Parallel (\(\Omega\)) & \\
    \hline
    \(R_{\text{total}}\) & 5463 & 6016.98 & 10.14 & 148.33 & 159.46 & 7.50 \\ % %9997 (LDR) %058204 %44209 %53698 (LDR) %7213952 (LDR)
    \hline\hline
  \end{tabular}
  \caption{Total resistance in series and parallel circuits}
\end{table}

The last part of the first procedure used the resistance value measured with the multimeter to estimate \(I_{\text{total}}\), which can be calculated using Ohm's Law (Equation 1).
\begin{equation}
  \begin{split}
    I_{\text{total}} &= \dfrac{V_{\text{total}}}{R_{\text{total}}} \\
  \end{split}
\end{equation}
\(R_{\text{total}}\) is simply the resistances measured by the multimeter summed together by Equation 2 (theoretical resistance in series in Table 2), while \(V_{\text{total}}\) is the average value from Figure 2, table c. Thus:
\begin{equation*}
  \begin{split}
    I_{\text{total}} &= \dfrac{10.63\text{ }\text{V}}{5463\text{ }\Omega} \\
    I_{\text{total}} &= 0.0019\text{ A} \\ %45817316
  \end{split}
\end{equation*}
The result yields a percentage error of 10.14\%. %058201

Finally, given the data for the last part of the lab using a combination of the first two circuits, the experimental current values can be compared to the theoretical values. First the total resistance is calculated:
\begin{equation}
  \begin{split}
    R_{\text{total}} &= R_{1} + \dfrac{1}{\tfrac{1}{R_{2}} + \tfrac{1}{R_{3}}} \\
  \end{split}
\end{equation}
Using Equation 7, the total resistance is calculated:
\begin{equation*}
  \begin{split}
    R_{\text{total}} &= 269 + \dfrac{1}{\tfrac{1}{549} + \tfrac{1}{1085}} \\
    R_{\text{total}} &= 633.54\text{ }\Omega \\ %40636
  \end{split}
\end{equation*}
Then the total current can be calculated using Ohm's Law (Equation 6), the total resistance, and the total voltage (from Figure 5, table c):
\begin{equation*}
  \begin{split}
    I_{\text{total}} &= \dfrac{10.58}{633.54} \\
    I_{\text{total}} &= 0.017\text{ A} \\ %69970663 (LDR)
  \end{split}
\end{equation*}
\(I_{\text{total}} = I_{1}\), giving the current through the first resistor. Since the experimental value for \(I_{1}\) was 0.0167, the error percentage was 4.19\%. For the second and third resistors, the following is known from Kirchhoff's First Law: %9933665 (LDR)
\begin{equation}
  \begin{split}
    I_{1} &= I_{2} + I_{3} \\
  \end{split}
\end{equation}
Additionally, voltage across resistors in parallel is equal, thus:
\begin{equation}
  \begin{split}
    V_{\text{total}} &= V_{1} + V_{2} = V_{1} + V_{3} \\
  \end{split}
\end{equation}
The voltage across resistor one, using Ohm's Law (Equation 1) is:
\begin{equation*}
  \begin{split}
    V_{1} &= (0.017\text{ A})(269\text{ }\Omega) \\
    V_{1} &= 4.49\text{ V} \\ %2221083
  \end{split}
\end{equation*}
Therefore, the voltage for each of the branches is as such:
\begin{equation*}
  \begin{split}
    V_{2} = V_{3} &= V_{\text{total}} - V_{1} \\
    &= 10.58 - 4.49 \\
    &= 6.09 \text{ V} \\ %7778917 (LDR)
  \end{split}
\end{equation*} 
Thus, \(I_{2}\) and \(I_{3}\) can now be calculated using Ohm's Law (Equation 6):
\begin{equation*}
  \begin{split}
    I_{2} &= \dfrac{6.09\text{ V}}{549\text{ }\Omega} \\
    I_{2} &= 0.011 \text{ A} \\ %08885049
  \end{split}
\end{equation*}
with an error percentage of 6.81\% as the experimental value was 0.0103. By the same process, \(I_{3}\) can be calculated: %3307571
\begin{equation*}
  \begin{split}
    I_{3} &= \dfrac{6.09\text{ V}}{1085\text{ }\Omega} \\
    I_{3} &= 0.0056 \text{ A} \\ %10856145
  \end{split}
\end{equation*}
Since \(I_{3}\) had an experimental value of 0.0051 A, the error percentage was 9.10\%. %4780657

\subsection* {DISCUSSION OF RESULTS}
Overall the error percentages stayed below 15\%, showing minimal deviation. Possible sources of error include the internal resistance in the wires and measuring devices, as well as the effect the meters (especially the multimeter) would have on the current. For instance, in lab 4, an electrometer was used to minimize current draw.

\subsection* {FURTHER STUDY}
If this experiment were to be replicated and precision warranted, the wire length should be minimized, as well as any outside influences on the current. Other experiments could also examine the relationship between different variables in a circuit using unique orientations, such as the Wheatstone bridge circuit.

\subsection* {REFERENCES}
Illinois Institute of Technology. (n.d.). Experiment 5: Simple Resistor Circuits. PDF. Chicago.
\end {document}
