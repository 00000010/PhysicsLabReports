% Document notes:
% Improvements:
%   - In Figure 3, make the total voltage table go directly underneath the current table and to the right of the resistor voltage table.
%   - There may be a bug with the circuitikz package where it should be able to put plus/minus signs with batteries (last heard of in 2014)
%   - Cannot draw voltage across resistors in parallel circuit? May be because they are vertical.
\documentclass [12pt, letterpaper, twoside] {article}
\usepackage[utf8]{inputenc}
\usepackage [left=1.0in, right=1.0in, top=1.0in, bottom=1.0in] {geometry}
% For updated time
\usepackage {datetime}
% For drawing pictures
\usepackage {tikz}
% For equations
\usepackage {amsmath}
% To make tables
\usepackage {tabu}
% For multiple rows in table slot
\usepackage {multirow}
\usepackage {verbatim}
% To add captions
\usepackage {caption}
\usepackage {float}
% To make graphs
\usepackage {pgfplots}
% To make scatter plots
\usepackage{pgfplotstable}
% To make pretty theorems
\usepackage[english]{babel}
% To make circuit drawings
\usepackage{circuitikz}
% To make tables side by side
\usepackage{subfig}

\usetikzlibrary {shapes.geometric, arrows, angles}
\newtheorem{theorem}{Theorem}

\tikzstyle {pink1circle0} = [circle, minimum size=0.5cm, text centered, draw=black, fill=pink1]
\tikzstyle {arrow} = [thick, ->, >=stealth]
\renewcommand {\labelitemiv}{$\triangle$}

\raggedbottom
\pgfplotsset{compat=1.16}
\begin {document}
\begin {titlepage}
\begin {center}
Department of Biological, Chemical, and Physical Science\\
\vspace {0.1cm}
Illinois Institute of Technology\\
\vspace {0.1cm}
General Physics II: Electromagnetism (PHYS 221-01)\\
\vspace* {\fill}
\begingroup
\Large
\textbf {Simple Resistor Circuits}
\vspace {0.35cm}

\normalsize
Lab 5 
\vspace {1.5cm}
\endgroup
\vspace* {\fill}
\end {center}

\vspace*{\fill}
\begin {flushright}
\footnotesize
Emily Pang, Lavanya Roy (lab partner) \\
Date of experiment: 26 Feb 2020 \\
Due date: 4 Mar 2020 \\
Lab section L06 \\
TA: Will Limestall \\
Updated \usdate\today~(\currenttime)
\end {flushright}
\end {titlepage}
\subsection* {STATEMENT OF OBJECTIVE}
The objective of this lab was to examine Ohm's Law, its connection with resistors, and how resistance changes in series, parallel, or both types of circuits.

\subsection* {THEORY}
\noindent
A circuit can be described by Ohm's Law, which is represented by the following:
\begin{equation}
  \begin{split}
    V &= IR \\
  \end{split}
\end{equation}
where \(V\) is the voltage measured in Volts (V), \(I\) is the current measured in Amps (A), and \(R\) is the resistance measured in Ohms (\(\Omega\)). Two other concepts will help describe circuits, namely Kirchhoff's First Law and Kirchhoff's Second Law.
\begin{theorem}[Kirchhoff's First Law]
  The current entering a junction is the same as the current leaving a junction.
\end{theorem}
\begin{theorem}[Kirchhoff's Second Law]
  The sum of all the voltage drops around a closed circuit is equal to zero.
\end{theorem}
Furthermore, resistance in a circuit is described differently depending on whether it is in series or parallel. If the resisters are in series, their resistances are added together.
\begin{equation}
  \begin{split}
    R_{\text{total}} &= R_{1} + R_{2} + R_{3} + \cdots \\
  \end{split}
\end{equation}
If the resistances are in parallel, the total resistance is the inverse of the addition of the inverses of all of the resistances.
\begin{equation}
  \begin{split}
    R_{\text{total}} &= \dfrac{1}{\tfrac{1}{R_{1}} + \tfrac{1}{R_{2}} + \tfrac{1}{R_{3}} + \cdots} \\
  \end{split}
\end{equation}
Current in a circuit will result in different values depending on the placement of the ammeter. The current in resistors in series is equal, while current in parallel splits at junctions by Kirchhoff's First Law. 

\subsection* {EQUIPMENT}
  \noindent
  \begin {itemize}
    \itemsep0em
    \item {four different resistors}
    \item {one multimeter}
    \item {one ammeter}
    \item {about 4 stripped wires}
    \item {power supply}
    \item {one breadboard}
    \item {wires to connect power supply and breadboard}
  \end {itemize}

\subsection* {PROCEDURE}
Each of the four resistors were first measured by the multimeter. Then all but the fourth resistor were constructed into a series circuit, as shown in Figure 1, diagram a. By Ohm's Law (Equation 1), resistance can be calculated if both the voltage across the resistor and the current through the resistor are known. Thus, the multimeter was used to measure the voltage and the ammeter was used to measure the current for each resistor. Ohm's Law (Equation 1) can again be used for finding the total resistance experimentally. As for resistor 4, it was added to the series circuit and its value was used to obtain an experimental value of the current.

\begin{figure}
  \centering
  \subfloat[Series circuit]{
    \begin{circuitikz}[]
      \draw (0,0) -- (0,2)
      to[R=\(R_{1}\), v_=\(V_{1}\)] (2,2) -- (2,2)
      to[R=\(R_{1}\), v_=\(V_{2}\)] (3,2) -- (3,2)
      to[R=\(R_{3}\), v_=\(V_{3}\)] (5,2) -- (5,0) to[battery1, l=\(V\), i=\(I\)] (0,0);
    \end{circuitikz}
  }\qquad
  \subfloat[Parallel circuit]{
    \begin{circuitikz}[american]
      \draw (4.5,0) -- (0,0) to[R=\(R_{1}\)] (0,3) -- (4.5,3) to[battery1, l=\(V\), i=\(I\)] (4.5,0);
      \draw (1.5,0) to[R=\(R_{2}\)] (1.5,3);
      \draw (3,0) to[R=\(R_{3}\)] (3,3);
    \end{circuitikz}
  }\qquad
  \subfloat[Combined circuit]{
    \begin{circuitikz}[]
      \draw (4,4.5) to[battery1, l=\(V\), i=\(I\)] (4,0) -- (0,0);
      \draw (4,4.5) -- (0,4.5) -- (0,4.5) to[R=\(R_{1}\)] (0,2.5);
      \draw (0,2.5) -- (-0.75,2.5) to[R=\(R_{2}\)] (-0.75,0.5) -- (0.75,0.5);
      \draw (0,2.5) -- (0.75,2.5) to[R=\(R_{3}\)] (0.75,0.5);
      \draw (0,0.5) -- (0,0);
    \end{circuitikz}
  }
  \caption{Circuit configurations}
\end{figure}

The second part of the experiment looked at resistors in parallel, and was constructed as in Figure 1, diagram b. Each resistor in the parallel circuit was measured, while the experimental total resistance was obtained using Ohm's Law (Equation 1) and the total measured current and voltage. Total resistance was then found again using Equation 3.

The last part of the experiment examined a combination of the first two parts and used the circuit in Figure 1, diagram c. In order to validate Kirchhoff's Laws, \(I_{1}\), \(I_{2}\), \(I_{3}\), \(V_{\text{total}}\), \(V_{1}\), \(V_{2}\), and \(V_{3}\) were measured and then compared to the expected values using Kirchhoff's Laws. 
    
\subsection* {DATA}
The resistances as measured by the multimeter and calculated for series and parallel circuits are shown in Table 1. The colors on each of the resistors are shown below: 

\vspace{0.5cm}
\indent
\textbf{\(R_{1}\):} Red, violet, brown, gold (\(27\times10^{1}\pm5\%\text{ }\Omega\))

\indent
\textbf{\(R_{2}\):} Green, blue, brown, gold (\(56\times10^{1}\pm5\%\text{ }\Omega\))

\indent
\textbf{\(R_{3}\):} Brown, brown, red, gold (\(11\times10^{2}\pm5\%\text{ }\Omega\))

\indent
\textbf{\(R_{4}\):} Orange, orange, red, silver (\(33\times10^{2}\pm10\%\text{ }\Omega\))
\vspace{0.5cm}

\begin{table}
  \centering
  \begin{tabular}{| l | r | c | c | c | c |}
    \hline\hline
    \multirow {3}{*}{Resistor} & & Calculated & \% & Calculated & \% \\
    & Resistance (\(\Omega\)) & Resistance, & Error & Resistance, & Error \\
    & & Series (\(\Omega\)) & & Parallel (\(\Omega\)) & \\
    \hline
    \multirow {3}{*}{\(R_{1}\)} & 270 & & & & \\
    & 268 & & & & \\
    & 269 & & & & \\
    \hline
    Average & 269 & 303.96 & 13.00 & 289.03 & 7.45 \\ %97112 (LDR) %617517 (LDR) %10786 %6497621 (LDR)
    \hline
    \multirow {3}{*}{\(R_{2}\)} & 549 & & & & \\
    & 549 & & & & \\
    & 549 & & & & \\
    \hline
    Average & 549 & 616.11 & 12.22 & 585.56 & 6.66 \\ %73412 (LDR) %355942 %5556 (LDR) %857122 (LDR)
    \hline
    \multirow {3}{*}{\(R_{3}\)} & 1085 & & & & \\
    & 1085 & & & & \\
    & 1085 & & & & \\
    \hline
    Average & 1085 & 1220.13 & 12.45 & 1188.7218 & 9.56 \\ %4146 %476 %9612903 (LDR)
    \hline
    \multirow {3}{*}{\(R_{4}\)} & 3560 & & & & \\
    & 3560 & & & & \\
    & 3560 & & & & \\
    \hline
    Average & 3560 & 3916.98 & 10.03 & 3809.64 & 7.01 \\ %0393 %753913 (LDR) %8095 (LDR) %2306039
    \hline\hline
  \end{tabular}
  \caption{Resistances by multimeter, in series, and in parallel circuits}
\end{table}

\noindent
The voltage across each of the resistors and the total current are shown in Figure 2. For the series circuit with \(R_{4}\), the voltage (\(V_{4}\)) and total current is again measured, as shown in Figure 3. Next, data was collected for the current running through each resister when all resistors were in a parallel circuit. The total voltage was measured again. These results are shown in Figure 4.

\begin{figure}
  \centering
  \subfloat[Voltage across each resistor]{
    \begin{tabular}{| l | r |}
      \hline\hline
      Voltage on Resistor & Voltage (V) \\
      \hline
      \multirow {3}{*}{\(V_{1}\)} & 1.509 \\
      & 1.510 \\
      & 1.510 \\
      \hline
      Average & 1.509666667 \\
      \hline
      \multirow {3}{*}{\(V_{2}\)} & 3.06 \\
      & 3.06 \\
      & 3.06 \\
      \hline
      Average & 3.06 \\
      \hline
      \multirow {3}{*}{\(V_{3}\)} & 6.06 \\
      & 6.06 \\
      & 6.06 \\
      \hline
      Average & 6.06 \\
      \hline\hline
    \end{tabular}
  }\qquad
  \subfloat[Total current]{
    \begin{tabular}{| l | r |}
      \hline\hline
      Current & Current (A) \\
      \hline
      \multirow {3}{*}{\(I_{\text{total}}\)} & 0.0050 \\
      & 0.0049 \\
      & 0.0050 \\
      \hline
      Average & 0.004966667 \\
      \hline\hline
    \end{tabular}
  }\qquad
  \subfloat[Total voltage]{
    \begin{tabular}{| l | r |}
      \hline\hline
      Voltage & Voltage (V) \\
      \hline
      \multirow {3}{*}{\(V_{\text{total}}\)} & 10.63 \\
      & 10.63 \\
      & 10.63 \\
      \hline
      Average & 10.63 \\
      \hline\hline
    \end{tabular}
  }
  \caption{Resistors in series without \(R_{4}\)}
\end{figure}

\begin{figure}
  \centering
  \subfloat[Voltage]{
    \begin{tabular}{| l | r |}
      \hline\hline
      Voltage on Resistor & Voltage (V) \\
      \hline
      \multirow {3}{*}{\(V_{4}\)} & 6.92 \\
      & 6.92 \\
      & 6.92 \\
      \hline
      Average & 6.92 \\
      \hline\hline
    \end{tabular}
  }\qquad
  \subfloat[Current]{
    \begin{tabular}{| l | r |}
      \hline\hline
      Total current & Current (A) \\
      \hline
      \multirow {3}{*}{\(I_{\text{total}}\)} & 0.0018 \\
      & 0.0017 \\
      & 0.0018 \\
      \hline
      Average & 0.001766667 \\
      \hline\hline
    \end{tabular}
  }
  \caption{Resistors in series with \(R_{4}\)}
\end{figure}
    
\begin{figure}
  \centering
  \subfloat[Current through each resister]{
    \begin{tabular}{| l | r |}
      \hline\hline
      Current through resistor & Current (A) \\
      \hline
      \multirow {3}{*}{\(I_{1}\)} & 0.0365 \\
      & 0.0365 \\
      & 0.0364 \\
      \hline
      Average & 0.03646666667 \\
      \hline
      \multirow {3}{*}{\(I_{2}\)} & 0.0180 \\
      & 0.0180 \\
      & 0.0180 \\
      \hline
      Average & 0.0180 \\
      \hline
      \multirow {3}{*}{\(I_{3}\)} & 0.0089 \\
      & 0.0089 \\
      & 0.0088 \\
      \hline
      Average & 0.0088666667 \\
      \hline
      \multirow {3}{*}{\(I_{4}\)} & 0.0027 \\
      & 0.0028 \\
      & 0.0028 \\
      \hline
      Average & 0.002766667 \\
      \hline\hline
    \end{tabular}
  }\qquad
  \subfloat[Total voltage]{
    \begin{tabular}{| l | r |}
      \hline\hline
      Voltage & Voltage (V) \\
      \hline
      \multirow {3}{*}{\(V_{\text{total}}\)} & 10.54 \\
      & 10.54 \\
      & 10.54 \\
      \hline
      Average & 10.54 \\
      \hline
    \end{tabular}
  }
  \caption{Resistors in parallel}
\end{figure}

\subsection* {ANALYSIS OF DATA}
In order to evaluate resistance in a series circuit, the experimental values obtained by measuring the voltage across each resistor and the total current were compared to the established resistance values using Equation 1. Since the current going through the resistors is identical in series circuits, each resistor value was found by the following:
\begin{equation*}
  \begin{split}
    R_{n} = \dfrac{V_{n}}{I_{\text{total}}} \\
  \end{split}
\end{equation*}
where \(R_{n}\) is the resistance of some resister \(n\) in a series circuit. The results of the calculations are shown in Table 1. For the parallel circuit, the individual resistances were calculated using Ohm's Law again (Equation 1) and the fact that voltage across resistors in parallel is equal. The results of these calculations are shown in Table 1. Overall, when calculating the resistor values whether in a series or parallel circuit, the values were relatively close to each other, as witnessed in Table 1. The highest error percentage was around 13\%, which is a relatively modest difference. %550843 (LDR) %9246945 (LDR)

For the total resistances in the series and parallel circuits, the theoretical values were caluclated using Equation 2 and 3, respectively. The experimental results for each circuit were measured by using Ohm's Law and Kirchhoff's Second Law for the series circuit, and Ohm's Law for the parallel circuit. For the series circuit, Kirchhoff's Second Law holds that:
\begin{equation}
  \begin{split}
    V_{\text{total}} &= V_{1} + V_{2} + V_{3} + \cdots + V_{n} \\
  \end{split}
\end{equation}
Then, Ohm's Law shows that:
\begin{equation}
  \begin{split}
    I_{\text{total}}R_{\text{total}} &= I_{1}R_{1} + I_{2}R_{2} + I_{3}R_{3} + \cdots + I_{n}R_{n} \\
    R_{\text{total}} &= \dfrac{I_{1}R_{1} + I_{2}R_{2} + I_{3}R_{3} + \cdots + I_{n}R_{n}}{I_{\text{total}}} \\
  \end{split}
\end{equation}
Since the current in a series circuit is equal through all the resistors, \(I_{\text{total}} = I_{1} = I_{2} = I_{3} = I_{n}\). The following is then given:
\begin{equation*}
  \begin{split}
    R_{\text{total}} &= R_{1} + R_{2} + R_{3} + \cdots + R_{n} \\
  \end{split}
\end{equation*}
Thus, the experimental total resistance is simply the experimental resistances found using Ohm's Law summed together. For the parallel circuit, the following was used:
\begin{equation*}
  \begin{split}
    R_{\text{total}} &= \dfrac{V_{\text{total}}}{I_{\text{total}}} \\
  \end{split}
\end{equation*}
since \(V_{\text{total}} = V_{1} = V_{2} = V_{3}\) in a series circuit, as in this experiment. The results for both calculations are shown in Table 2. Error percentages show that the results are consistent with the values obtained theoretically, validating Ohms's Law and the two equations for calculating resistances in series and parallel circuits.

The last part of the first procedure used the resistance value measured with the multimeter to estimate \(I_{\text{total}}\), which can be calculated using Ohm's Law (Equation 1).
\begin{equation}
  \begin{split}
    I_{\text{total}} &= \dfrac{V_{\text{total}}}{R_{\text{total}}} \\
  \end{split}
\end{equation}
\(R_{\text{total}}\) is simply the resistances measured by the multimeter summed together by Equation 2 (theoretical resistance in series in Table 2), while \(V_{\text{total}}\) is the average value from Figure 2, table c. Thus:
\begin{equation*}
  \begin{split}
    I_{\text{total}} &= \dfrac{10.63\text{ }\text{V}}{5463\text{ }\Omega} \\
    I_{\text{total}} &= 0.001945817316 \\
  \end{split}
\end{equation*}
The result is 

\begin{table}
  \centering
  \begin{tabular}{| l | c | c | c | c | c | c |}
    \hline\hline
    \multirow {3}{*}{Resistor} & Theoretical & Experimental & \% & Theoretical & Experimental & \% \\
    & Resistance, & Resistance, & Error & Resistance & Resistance, & Error \\
    & Series (\(\Omega\)) & Series (\(\Omega\)) & & Parallel (\(\Omega\)) & Parallel (\(\Omega\)) & \\
    \hline
    \(R_{\text{total}}\) & 5463 & 6016.98 & 10.14 & 148.33 & 159.46 & 7.50 \\ % %9997 (LDR) %058204 %44209 %53698 (LDR) %7213952 (LDR)
    \hline\hline
  \end{tabular}
  \caption{Total resistance in series and parallel circuits}
\end{table}

\pgfplotstableread{
X Y
0.056166667 0.00000000659727543
0.006666667 0.000000000263492357
0.064       0.00000000808568429
0.049333333 0.00000000540002727

0.036333333 0.00000000338413143
0.009       0.000000000413412471
0.057333333 0.00000000681107479
0.065       0.00000000828470485

0.060666667 0.00000000743708455
0.066333333 0.000000000484261111
0.057666667 0.00000000391738668
0.066       0.00000000987984447

0.034       0.00000000305966295
0.01        0.000000000484261111
0.04        0.00000000391738668
0.072666667 0.00000000987984447
}\distanceAndCharge

\begin {figure}
  \centering
  \begin{tikzpicture}
    \begin{axis}[
      title = {Distance Between Charges and Charge Values},
      xlabel = {\(r\) (m)},
      ylabel = {\(q\) (C)},
      ]
      \addplot [only marks, mark = *] table {\distanceAndCharge};
      \addplot [thick, red] table[
        y={create col/linear regression={y=Y}}
      ]
      {\distanceAndCharge};
    \end{axis}
  \end{tikzpicture}
  \caption {}
\end {figure}

\subsection* {DISCUSSION OF RESULTS}

\subsection* {FURTHER STUDY}

\subsection* {REFERENCES}
Illinois Institute of Technology. (n.d.). Experiment 5: Simple Resistor Circuits. PDF. Chicago.
\end {document}
